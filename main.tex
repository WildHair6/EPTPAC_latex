\documentclass[journal]{new-aiaa}
%\documentclass[conf]{new-aiaa} for conference papers
\usepackage[utf8]{inputenc}
\usepackage{textcomp}

\usepackage{graphicx}
\usepackage{amsmath}
\let\openbox\undefined
\usepackage{amsthm}    % for theorem, lemma environments
% \usepackage{thmtools}   % to avoid \openbox conflict
\usepackage[version=4]{mhchem}
\usepackage{siunitx}
\usepackage{longtable,tabularx}
\usepackage{booktabs}
\usepackage{subfig} % 必须包含此包才能使用 \subfloat
% \renewcommand{\includegraphics}[2][]{\fbox{Image: #2}}


\setlength\LTleft{0pt} 

\newtheorem{theorem}{Theorem}
\newtheorem{lemma}{Lemma}
\newtheorem{remark}{Remark}
\newtheorem{assumption}{Assumption}



\title{Exact Prescribed-Time Prescribed-Accuracy Control for Spacecraft under Dual Reaction Wheel Saturation}

\author{First A. Author \footnote{Insert Job Title, Department Name, Address/Mail Stop, and AIAA Member Grade (if any) for first author.} and Second B. Author Jr.\footnote{Insert Job Title, Department Name, Address/Mail Stop, and AIAA Member Grade (if any) for second author.}}
\affil{Business or Academic Affiliation 1, City, State, Zip Code}
\author{Third C. Author\footnote{Insert Job Title, Department Name, Address/Mail Stop, and AIAA Member Grade (if any) for third author.}}
\affil{Business or Academic Affiliation 2, City, Province, Zip Code, Country}
\author{Fourth D. Author\footnote{Insert Job Title, Department Name, Address/Mail Stop, and AIAA Member Grade (if any) for fourth author (etc.).}}
\affil{Business or Academic Affiliation 2, City, State, Zip Code}

\begin{document}

\maketitle

\begin{abstract}
This paper proposes an Exact Prescribed-Time Prescribed-Accuracy Control (EPTPAC) framework for rigid 
spacecraft equipped with reaction wheels subject to dual saturation, 
which includes the torque saturation and the wheel angular-momentum saturation.
The proposed scheme guarantees that attitude tracking errors 
converge to a user-specified accuracy neighborhood exactly at a user-defined terminal time \(T_f\), 
while strictly respecting the dual saturation constraints throughout the maneuver. The approach employs 
a two-loop architecture: an outer-loop trajectory planner generates a dynamically feasible reference that 
satisfies the dual saturation bounds and reaches the target state exactly at \(T_f\); an inner-loop nonsingular 
prescribed-time tracking controller then drives the tracking errors into the prescribed accuracy neighborhood 
before \(T_f\) with bounded control gain. A systematic parameter-synthesis procedure 
is also developed to coordinate the two loops and ensure closed-loop feasibility under bounded disturbances. 
Numerical simulations of large-angle maneuvers demonstrate exact terminal-time convergence, 
robustness, and strict compliance with dual saturation constraints.
\end{abstract}

\section*{Nomenclature}
{\renewcommand\arraystretch{1.0}
\noindent\begin{longtable}{@{}l@{\quad}l@{}}
\multicolumn{2}{@{}l}{\textbf{Roman Symbols}} \\

$\boldsymbol{d}$ & lumped disturbance torque [\si{N\cdot m}] \\
$d_{\max}$ & bound on $\|\boldsymbol{d}(t)\|$ [\si{N\cdot m}] \\

$\boldsymbol{h}_w$ & reaction-wheel angular momentum (body frame) [\si{kg\cdot m^2/s}] \\
$h_{w,i,\max}$ & maximum allowable wheel momentum (axis $i$) [\si{kg\cdot m^2/s}] \\
$\boldsymbol{h}_{w,0}$ & initial wheel momentum [\si{kg\cdot m^2/s}] \\

$\boldsymbol{H}$ & total angular momentum, $\boldsymbol{H}=\boldsymbol{J}\boldsymbol{\omega}+\boldsymbol{h}_w$ [\si{kg\cdot m^2/s}] \\
$H_{\max}$ & bound on $\|\boldsymbol{H}(t)\|$ over $[0,T_f]$ [\si{kg\cdot m^2/s}] \\

$\boldsymbol{J}$ & spacecraft inertia matrix [\si{kg\cdot m^2}] \\
$\lambda_{\min}(\boldsymbol{J}),\,\lambda_{\max}(\boldsymbol{J})$ & minimum/maximum eigenvalues of $\boldsymbol{J}$ [\si{kg\cdot m^2}] \\

$\boldsymbol{R}(\boldsymbol{\sigma}_e)$ & direction cosine matrix associated with $\boldsymbol{\sigma}_e$ [--] \\
$\boldsymbol{G}(\boldsymbol{\sigma})$ & MRP kinematic matrix [--] \\

$T_f$ & user-specified terminal time [\si{s}] \\
$T_{p1},\,T_{p2}$ & prescribed convergence times (attitude/sliding variable) [\si{s}] \\

$\boldsymbol{\tau}$ & control torque applied to spacecraft body [\si{N\cdot m}] \\
$\tau_{i,\max}$ & maximum allowable torque (axis $i$) [\si{N\cdot m}] \\
$\boldsymbol{\tau}_{\mathrm{ref}}$ & planned reference torque from the outer-loop OCP [\si{N\cdot m}] \\

$\boldsymbol{\omega}$ & spacecraft angular velocity (body frame) [\si{rad/s}] \\
$\boldsymbol{\omega}_d$ & reference angular velocity [\si{rad/s}] \\
$\boldsymbol{\omega}_e$ & angular-velocity tracking error [\si{rad/s}] \\
$\omega_{d,\max}$ & bound on $\|\boldsymbol{\omega}_d(t)\|$ on $[0,T_f]$ [\si{rad/s}] \\
$\dot{\omega}_{d,\max}$ & bound on $\|\dot{\boldsymbol{\omega}}_d(t)\|$ on $[0,T_f]$ [\si{rad/s^2}] \\

$\boldsymbol{\sigma}$ & Modified Rodrigues Parameters (MRPs) [--] \\
$\boldsymbol{\sigma}_d$ & reference MRPs [--] \\
$\boldsymbol{\sigma}_e$ & attitude tracking error in MRPs [--] \\

$\mu_i$ & row-sum coefficient of inertia, $\mu_i=\sum_{j=1}^3 |J_{ij}|$ [\si{kg\cdot m^2}] \\
$\lambda$ & OCP smoothness weight in \eqref{eq:OCP_slack} [--] \\

\multicolumn{2}{@{}l}{\textbf{Greek Symbols}} \\

$\eta$ & exponent in prescribed-time law, $0<\eta<1$ [--] \\

$\varepsilon_1,\,\varepsilon_2$ & prescribed accuracy tolerances (attitude/sliding variable) [--] \\
$\varepsilon_{\text{mission}}$ & mission-level attitude accuracy requirement [--] \\

$\alpha_1,\,\alpha_2$ & tunable gains in prescribed-time laws [--] \\

$\gamma_u,\,\gamma_h$ & constraint-tightening margins (torque/momentum) [--] \\

\end{longtable}}



\section{Introduction}

Time-critical spacecraft missions, such as rendezvous, docking, and formation reconfiguration, 
require the attitude to reach a commanded state at an exact user-specified terminal time $T_f$, 
which is dictated by orbital geometry, communication windows, or multi-vehicle coordination 
requirements \cite{alex_pothen_pose_2022,sun_attitude_2024,xu_distributed_2024}. This terminal-time requirement 
becomes substantially more challenging for reaction-wheel spacecraft that operate under dual 
reaction-wheel saturation, which includes the torque saturation and the wheel angular-momentum 
saturation.Hereafter, dual saturation denotes the joint torque and momentum saturation of reaction wheels
During large-angle maneuvers, wheel momentum can saturate 
even when torque commands stay within bounds, which makes many analytically valid controllers 
physically unrealizable. Accordingly, a practical terminal-time control strategy should guarantee 
arrival at $t=T_f$ while strictly respecting dual saturation throughout the maneuver.

To overcome the sluggishness of asymptotic designs, finite-time control (FTC) 
\cite{hong_finite-time_2006,zhou_finite-time_2020} and fixed-time control (FxTC) 
\cite{polyakov_nonlinear_2012,polyakov_finite-time_2015} have been widely studied. 
FTC guarantees convergence in finite time, but the settling time depends on initial conditions, 
which weakens timing predictability. FxTC removes this dependence, but it typically provides a 
conservative upper bound that cannot be aligned with a mission-specified terminal time. 
These limitations motivate prescribed-time control, where the convergence 
time enters explicitly as a design parameter.

Existing prescribed-time methods can be broadly categorized into three lines. The first embeds 
the prescribed time into Lyapunov inequalities using exponential and power terms 
\cite{sanchez-torres_predefined-time_2015,sanchez-torres_class_2018} and has been applied to robotic tracking, 
launch-vehicle attitude control, flexible spacecraft, and formation flying 
\cite{anguiano-gijon_predefined-time_2019,ye_predefined-time_2022,munoz-vazquez_predefined-time_2019,meng_predefined-time_2025,xiao_predefined-time_2025,xu_distributed_2022,xu_distributed_2024}. 
These designs are markedly conservative, often driving convergence well before the specified time and inducing very 
large initial torque demands. The second line employs time-varying high-gain feedback with time-warping transformations 
\cite{song_time-varying_2017,zhou_finite-time_2020} and has been used in consensus and robust tracking 
\cite{wang_prescribed-time_2019,cao_practical_2022,zhang_global_2024,zhao_exponential_2024}. 
Compared with the first line, conservatism is reduced, but the gain typically grows 
rapidly as time approaches $T_f$, which makes the controller vulnerable to saturation and noise. In practice, 
truncation or smoothing is introduced to keep gains bounded \cite{xiao_scaling-transformation-based_2025,cao_practical_2022}, 
but this comes at the cost of performance, since terminal-time tightness and tracking accuracy are generally weakened. 
The third line leverages periodic delayed feedback to obtain smooth prescribed-time stabilization 
\cite{zhou_fixed-time_2022,ding_strong_2024,zhou_prescribed_2024}. Although it avoids singularities, 
it can be sensitive to disturbances and may induce inefficient attitude reversals, which increases momentum usage.
Despite these advances, existing PTC approaches for reaction-wheel spacecraft still exhibit two fundamental 
shortcomings that hinder their use in time-critical missions.

First, a number of PTC designs are conservative\cite{cao_practical_2022,song_time-varying_2017,xu_distributed_2024}.
They guarantee convergence no later than prescribed time $T_f$, 
yet the closed-loop response often settles significantly earlier,
as reported in \cite{xu_distributed_2024} where the prescribed time is set to $500~\mathrm{s}$ but 
the convergence occurs at about $250~\mathrm{s}$. 
Such premature convergence is unacceptable in missions with hard terminal-time constraints. 
For example, in on-orbit assembly or formation reconfiguration scenarios, arriving much ahead of schedule 
can result in task conflicts or physical interference with other vehicles that are scheduled 
to operate in the same workspace. Early settling also 
induces unnecessarily large torque transients and wastes momentum resources, which is particularly 
undesirable for reaction-wheel spacecraft.

Second, physical realizability under dual saturation remains insufficiently addressed. 
Although torque saturation has been incorporated into certain PTC designs 
\cite{sun_attitude_2024,xu_distributed_2022,zhao_exponential_2024}, the joint treatment of dual
saturation is rarely considered. 
More critically, the prescribed terminal time $T_f$ is often selected as a free tuning parameter, 
ignoring that it is fundamentally lower-bounded by the actuator's torque and momentum capacities 
together with the maneuver's kinematic demands. For large-angle attitude reconfigurations, this 
minimum feasible time can be substantial; prescribing a shorter horizon may render the control law 
infeasible or cause hidden constraint violations, even if the nominal design appears mathematically valid.

Optimization-based trajectory planning can provide a potential solution because it 
can explicitly embed torque and momentum limits and enforce terminal-time boundary 
conditions by construction 
\cite{kobayashi_optimal_2017,spiller_inverse_2018,celani_spacecraft_2020,tao_gradient-based_2025}. 
However, these approaches are applied in an open-loop manner. Under disturbances or modeling 
errors, tracking can drift away from the planned trajectory and terminal-time accuracy is not 
guaranteed. Some studies have combined optimization with feedback control to improve practical 
performance \cite{chai_six-dof_2020,gong_lyapunov-based_2021}, but aggressive maneuvers and 
significant disturbances can still lead to noticeable deviation, and terminal-time tracking 
guarantees remain largely unavailable.

Motivated by these observations, this paper proposes a EPTPAC framework built on a two-loop architecture. 
The outer loop generates a physically realizable reference trajectory that respects dual saturation and reaches the target state exactly at the user-specified terminal time $T_f$. 
The inner loop designs a nonsingular prescribed-time sliding-mode tracking controller, which guarantees that the spacecraft tracks the planned trajectory and drives the attitude and angular-velocity errors into a user-prescribed accuracy neighborhood before $T_f$, without gain blow-up. Together, the two loops ensure terminal-time attitude maneuvers that satisfy dual saturation while achieving user-specified accuracy at the mission time.
In addition, a systematic parameter-synthesis procedure is provided to coordinate the two loops and maintain feasibility in the presence of disturbances.

The main contributions of this paper are summarized as follows:
\begin{enumerate}
  \item  The framework guarantees that attitude tracking errors enter a user-specified accuracy neighborhood exactly at the prescribed terminal time $T_f$, eliminating premature convergence.
  \item The framework guarantees physical realizability under dual saturation, ensuring the prescribed terminal time $T_f$
  is actually achievable.
  \item The framework establishes a new prescribed-time control paradigm based on an extensible two-loop architecture 
  that jointly ensures trajectory feasibility and prescribed-time tracking performance.
\end{enumerate}

The remainder of this paper is organized as follows. 
Section~\ref{section:PROBLEM} formulates the spacecraft attitude dynamics with reaction wheels and 
the dual saturation constraints, and it reviews practical prescribed-time stability tools. 
Section~\ref{section:MAINRES} presents the proposed framework, including trajectory planning, 
controller design, and synthesis rules. Section~\ref{section:simulation} reports simulation studies 
for large-angle maneuvers under significant disturbances. Section~\ref{section:conclusion} concludes 
the paper and discusses future work.


\section{Problem Description and Preliminaries}
\label{section:PROBLEM}

\subsection{Problem Description}
Let $\mathbb{R}$ denote the set of real numbers. For vectors and matrices, $(\cdot)^T$ denotes transpose,
$\|\cdot\|$ denotes the Euclidean norm, and $\boldsymbol{I}_n$ denotes the $n\times n$ identity matrix.
For a symmetric positive definite matrix $\boldsymbol{A}$, $\lambda_{\min}(\boldsymbol{A})$ and
$\lambda_{\max}(\boldsymbol{A})$ denote its minimum and maximum eigenvalues, respectively.
For any $\boldsymbol{a}=[a_1,a_2,a_3]^T\in\mathbb{R}^3$, define
\[
\boldsymbol{a}^{\times} \triangleq
\begin{bmatrix}
0 & -a_3 & a_2\\
a_3 & 0 & -a_1\\
-a_2 & a_1 & 0
\end{bmatrix}
\]


The attitude kinematics and dynamics of a rigid spacecraft with reaction wheels can be written as
\begin{equation}
\dot{\boldsymbol{\sigma}} = \boldsymbol{G}(\boldsymbol{\sigma}) \boldsymbol{\omega}, 
\quad \boldsymbol{G}(\boldsymbol{\sigma}) = \frac{1}{4} \left[ \big(1 - \|\boldsymbol{\sigma}\|^2 \big) \boldsymbol{I}_3 + 2 {\boldsymbol{\sigma}}^{\times} + 2 \boldsymbol{\sigma} \boldsymbol{\sigma}^T \right]
\label{eq:attitude_kinematics}
\end{equation}
\begin{equation}
\boldsymbol{J} \dot{\boldsymbol{\omega}} + \boldsymbol{\omega}^{\times} (\boldsymbol{J} \boldsymbol{\omega} 
+ \boldsymbol{h}_w) = \boldsymbol{d} + \boldsymbol{\tau}, \quad \boldsymbol{\tau} = -\dot{\boldsymbol{h}}_w
\label{eq:attitude_dynamics}
\end{equation}
where $\boldsymbol{\sigma}\in\mathbb{R}^3$ denotes the modified Rodrigues parameters (MRPs) of the spacecraft attitude,
$\boldsymbol{\omega}\in\mathbb{R}^3$ is the body angular velocity with respect to the inertial frame expressed in the body frame,
$\boldsymbol{J}\in\mathbb{R}^{3\times 3}$ is the constant symmetric positive definite inertia matrix,
$\boldsymbol{h}_w\in\mathbb{R}^3$ is the reaction-wheel angular momentum expressed in the body frame,
$\boldsymbol{\tau}\in\mathbb{R}^3$ is the control torque applied to the spacecraft body,
and $\boldsymbol{d}\in\mathbb{R}^3$ denotes the lumped disturbance torque.


For clarity of exposition, a three-axis reaction-wheel cluster is considered, where the wheel axes are aligned with the body principal axes.
The reaction-wheel actuators are subject to dual saturation.
For each axis $i\in\{1,2,3\}$, the following constraints hold:
\begin{equation}
|\tau_i(t)| \le \tau_{i,\max}
\label{eq:torque_saturation}
\end{equation}
\begin{equation}
|h_{w,i}(t)| \le h_{w,i,\max}
\label{eq:momentum_saturation}
\end{equation}
where $\tau_i$ and $h_{w,i}$ denote the $i$th components of the control torque $\boldsymbol{\tau}$
and wheel angular momentum $\boldsymbol{h}_w$, respectively, and $\tau_{i,\max}>0$ and $h_{w,i,\max}>0$
are the corresponding component-wise bounds.

Let $\boldsymbol{\sigma}_d\in\mathbb{R}^3$ and $\boldsymbol{\omega}_d \in\mathbb{R}^3$ denote the desired attitude MRP
and angular velocity, respectively. The attitude tracking error $\boldsymbol{\sigma}_e\in\mathbb{R}^3$ is defined as the MRP
representation of the relative rotation from the desired frame to the current body frame:
\begin{equation}
\boldsymbol{\sigma}_e =
\frac{(1-\|\boldsymbol{\sigma}_d\|^2)\boldsymbol{\sigma}-(1-\|\boldsymbol{\sigma}\|^2)\boldsymbol{\sigma}_d
+2\boldsymbol{\sigma}_d^{\times}\boldsymbol{\sigma}}
{1+\|\boldsymbol{\sigma}\|^2\|\boldsymbol{\sigma}_d\|^2+2\boldsymbol{\sigma}^T\boldsymbol{\sigma}_d}
\end{equation}

\begin{equation}
\boldsymbol{\omega}_e = \boldsymbol{\omega} - \boldsymbol{R}(\boldsymbol{\sigma}_e)\boldsymbol{\omega}_d
\label{eq:angular_velocity_error_definition}
\end{equation}
where $\boldsymbol{R}(\boldsymbol{\sigma}_e) \in \mathbb{R}^{3\times 3}$ is the direction cosine matrix associated with $\boldsymbol{\sigma}_e$, given by
\begin{equation}
\boldsymbol{R}(\boldsymbol{\sigma}_e) = \boldsymbol{I}_3 + \frac{8 {\boldsymbol{\sigma}_e}^{\times} 
{\boldsymbol{\sigma}_e}^{\times} - 4 \big(1 - \boldsymbol{\sigma}_e^T \boldsymbol{\sigma}_e\big) 
{\boldsymbol{\sigma}_e}^{\times} }{\big(1 + \boldsymbol{\sigma}_e^T \boldsymbol{\sigma}_e\big)^2}
\label{eq:R_sigma_e_definition}
\end{equation}
To avoid the MRP singularity at $360^\circ$, the standard shadow-set switching can be applied;
hence, the error representation can be kept in the practical region $\|\boldsymbol{\sigma}_e(t)\|<1$.

Based on the above definitions, the tracking error kinematics and dynamics can be written as
\begin{equation}
\dot{\boldsymbol{\sigma}}_e = \boldsymbol{G}(\boldsymbol{\sigma}_e)\boldsymbol{\omega}_e , \quad
\boldsymbol{G}(\boldsymbol{\sigma}_e) = \frac{1}{4} \left[ \big(1 - \|\boldsymbol{\sigma}_e\|^2\big) \boldsymbol{I}_3 + 2 {\boldsymbol{\sigma}_e}^{\times} + 2 \boldsymbol{\sigma}_e \boldsymbol{\sigma}_e^T \right]
\label{eq:mrp_error_kinematics}
\end{equation}
\begin{equation}
\boldsymbol{J}\dot{\boldsymbol{\omega}}_e = \boldsymbol{\tau} + \boldsymbol{d} + \boldsymbol{f}
\label{eq:simplified_error_dynamics}
\end{equation}
where the 2-norm of the kinematic matrix $\boldsymbol{G}(\boldsymbol{\sigma}_e)$ satisfies
$\|\boldsymbol{G}(\boldsymbol{\sigma}_e)\| = \frac{1 + \|\boldsymbol{\sigma}_e\|^2}{4}$.
Consequently, for all $\|\boldsymbol{\sigma}_e\| < 1$, it follows that
\begin{equation}
\frac{1}{4} \le \|\boldsymbol{G}(\boldsymbol{\sigma}_e)\| < \frac{1}{2}.
\label{eq:G_norm_bound}
\end{equation}
Here, $\boldsymbol{f}\in\mathbb{R}^3$ is a known feedforward term determined by the measured states
and the reference signals, given by
\[
\boldsymbol{f} := - \boldsymbol{\omega}^{\times} (\boldsymbol{J} \boldsymbol{\omega} + \boldsymbol{h}_w)+
\boldsymbol{J} \boldsymbol{\omega}_e^{\times} \boldsymbol{R}(\boldsymbol{\sigma}_e)\boldsymbol{\omega}_d+
\boldsymbol{J} \boldsymbol{R}(\boldsymbol{\sigma}_e)\dot{\boldsymbol{\omega}}_d.
\]


The following assumptions are imposed.


\begin{assumption}\label{ass:feasible_Tf}
The user-specified terminal time $T_f>0$ satisfies
\begin{equation}
T_f \ge T_{f,\min}
\label{eq:T_f_feasible}
\end{equation}
where $T_{f,\min}>0$ is the minimum maneuver time required to transfer the spacecraft from the initial state
to the target state under the component-wise constraints \eqref{eq:torque_saturation}--\eqref{eq:momentum_saturation}.
Consequently, there exists a dynamically feasible reference trajectory $(\boldsymbol{\sigma}_d,\boldsymbol{\omega}_d)$ on $[0,T_f]$
with bounded angular velocity and angular acceleration: there exist known constants $\omega_{d,\max}>0$ and
$\dot{\omega}_{d,\max}>0$ such that
\begin{equation}
\|\boldsymbol{\omega}_d(t)\| \le \omega_{d,\max}, \qquad
\|\dot{\boldsymbol{\omega}}_d(t)\| \le \dot{\omega}_{d,\max},\quad \forall t\in[0,T_f]
\label{eq:ref_bounded}
\end{equation}
\end{assumption}

\begin{assumption}\label{ass:disturbance_bound}
The lumped disturbance torque $\boldsymbol{d}$ is bounded. Namely, there exists a known constant $d_{\max}>0$ such that
\begin{equation}
\|\boldsymbol{d}(t)\| \le d_{\max}, \quad \forall t\in[0,T_f]
\label{eq:disturbance_bound}
\end{equation}
\end{assumption}

\begin{assumption}\label{ass:H_budget}
Let $\boldsymbol{H}(t):=\boldsymbol{J}\boldsymbol{\omega}(t)+\boldsymbol{h}_w(t)$ denote the total angular momentum in the body frame.
The total angular momentum satisfies
\begin{equation}
\|\boldsymbol{H}(t)\| \le H_0 + H_{\max}, \qquad \forall t \in [0, T_f],
\label{eq:Hmax}
\end{equation}
where $H_0 = \|\boldsymbol{H}(0)\| \ge 0$ is the known initial total angular momentum,
and $H_{\max} > 0$ is a known bound on the momentum drift induced by external disturbances over $[0,T_f]$.
\end{assumption}

The control objective is to design a EPTPAC scheme for a given terminal time $T_f>0$ and a mission-required accuracy
$\varepsilon_{\mathrm{mission}}>0$ such that:
(i) the dual saturation constraints \eqref{eq:torque_saturation}--\eqref{eq:momentum_saturation} hold for all $t\in[0,T_f]$;
(ii) the spacecraft reaches the desired terminal state at $t=T_f$ by tracking the planned reference;
and (iii) the tracking errors $(\boldsymbol{\sigma}_e,\boldsymbol{\omega}_e)$ enter a neighborhood of the origin with radius
no larger than $\varepsilon_{\mathrm{mission}}$ before $T_f$ and remain bounded thereafter. The overall framework is shown in Fig.~\ref{main_work}.

\begin{figure}[htbp]
    \centering
    \includegraphics[width=0.5\linewidth]{fig/mainwork.pdf}
    \caption{The EPTPAC framework}
    \label{main_work}
\end{figure}

\subsection{Preliminaries}

This subsection summarizes basic tools on prescribed-time stability that will be used in the 
subsequent analysis and controller design. These results are standard and are included only for 
completeness.

Consider the nonlinear system
\begin{equation}
\dot{\boldsymbol{x}} = \boldsymbol{g}(\boldsymbol{x},t)
\label{eq:system}
\end{equation}
where $\boldsymbol{x}\in\mathbb{R}^n$ and $\boldsymbol{g}(\boldsymbol{x},t)$ is locally Lipschitz 
in $\boldsymbol{x}$, with $\boldsymbol{g}(\boldsymbol{0},t)=\boldsymbol{0}$. Given a constant $T_p>0$, 
the origin is said to be prescribed-time stable if, for any initial condition 
$\boldsymbol{x}(0)=\boldsymbol{x}_0$, the solution exists and reaches the origin in finite time 
$T(\boldsymbol{x}_0)\le T_p$.


\begin{lemma}[\cite{anguiano-gijon_predefined-time_2019}]\label{lem:PT}
Suppose there exists a Lyapunov function $V(\boldsymbol{x})$ of system \eqref{eq:system} such that
\begin{equation}
\dot V \le -\frac{\pi}{\eta T_p}\left(V^{1-\tfrac{\eta}{2}}+V^{1+\tfrac{\eta}{2}}\right)
\label{eq:lemma_PT_cond}
\end{equation}
where $0<\eta<1$ and $T_p>0$ are given constants. Then the state converges to the origin within time $T_p$, i.e., the origin is prescribed-time stable.
\end{lemma}

\begin{theorem}\label{thm:practical_PT}
Suppose there exists a Lyapunov function $V(\boldsymbol{x})$ for system \eqref{eq:system} such that
\begin{equation}
\dot V \le \delta V^{\tfrac{1}{2}} - \frac{\pi}{\eta T_p}\left[(1+\alpha) V^{1-\tfrac{\eta}{2}}+V^{1+\tfrac{\eta}{2}}\right]
\label{eq:theorem1_condition}
\end{equation}
where $\delta>0$, $\alpha>0$, $0<\eta<1$, and $T_p>0$ are constants. Then, for any initial condition, the trajectory enters the set
\begin{equation}
V \le V^*_\delta,\qquad 
V^*_\delta := \left(\frac{\delta\,\eta\,T_p}{\pi\,\alpha}\right)^{\frac{2}{1-\eta}}
\label{eq:V_star_delta}
\end{equation}
in no more than $T_p$ seconds and remains in it thereafter.
\end{theorem}


\noindent\textbf{Proof:}
Let $K:=\frac{\pi}{\eta T_p}>0$ and let $V^*_\delta$ be defined in \eqref{eq:V_star_delta}. From \eqref{eq:theorem1_condition},
\begin{equation}
\dot V \le \delta V^{\frac{1}{2}} - K\left[(1+\alpha)V^{1-\frac{\eta}{2}} + V^{1+\frac{\eta}{2}}\right]
\end{equation}
By construction, $V=V^*_\delta$ satisfies $\delta V^{1/2}=K\alpha V^{1-\eta/2}$. Moreover, for all $V\ge V^*_\delta$,
the function $V^{-(1-\eta)/2}$ is decreasing, hence $\delta V^{1/2}\le K\alpha V^{1-\eta/2}$, and therefore
\begin{equation}
\dot V \le -K\left(V^{1-\tfrac{\eta}{2}}+V^{1+\tfrac{\eta}{2}}\right), \qquad \forall\, V\ge V^*_\delta.
\end{equation}
By Lemma~\ref{lem:PT}, any trajectory with $V(0)>V^*_\delta$ reaches the set $\{V\le V^*_\delta\}$ within at most $T_p$.
Finally, note that at the boundary $V=V^*_\delta$ we have $\dot V\le -K\left(V^{1-\tfrac{\eta}{2}}+V^{1+\tfrac{\eta}{2}}\right)<0$,
which implies that the set $\{V\le V^*_\delta\}$ is forward invariant. This completes the proof. \hfill $\square$

\begin{lemma}\label{lem:norm_ineq}
For any vector $\boldsymbol{x}\in\mathbb{R}^n$ satisfying $\|\boldsymbol{x}\|\ge\varepsilon$ with $\varepsilon>0$, and any $0<\eta<1$, the following inequalities hold:
\begin{equation}
\frac{\|\boldsymbol{x}\|^2}{(\|\boldsymbol{x}\|+\varepsilon)^\eta} \ge \frac{1}{2^\eta} \|\boldsymbol{x}\|^{2-\eta}, \quad
\frac{\|\boldsymbol{x}\|^2}{(\|\boldsymbol{x}\|+\varepsilon)^{-\eta}} \ge \|\boldsymbol{x}\|^{2+\eta}
\end{equation}
\end{lemma}


\section{Main Results}
\label{section:MAINRES}

\subsection{Outer-Loop Trajectory Planning Under Dual Saturation}

To enforce exact arrival at the user-specified terminal time $T_f$, an offline reference
trajectory $(\boldsymbol{\sigma}_d(t),\boldsymbol{\omega}_d(t))$ is constructed by solving a constrained
optimal control problem (OCP) over $[0,T_f]$. The OCP explicitly embeds the spacecraft kinematics and
dynamics \eqref{eq:attitude_kinematics}--\eqref{eq:attitude_dynamics} together with the dual saturation
constraints \eqref{eq:torque_saturation}--\eqref{eq:momentum_saturation}, thereby ensuring that the
reference is dynamically consistent and actuator-feasible.

Specifically, the OCP is formulated as
\begin{equation}
\begin{aligned}
\min_{\boldsymbol{\tau}_{\mathrm{ref}}(\cdot),\,\Delta\boldsymbol{\sigma}_0} \quad &
J \;=\; \int_0^{T_f} \left( \|\boldsymbol{\tau}_{\mathrm{ref}}(t)\|^2
+ \lambda \|\dot{\boldsymbol{\tau}}_{\mathrm{ref}}(t)\|^2 \right)\, dt \\[2pt]
\text{s.t.}\quad
& \dot{\boldsymbol{\sigma}}(t) = \boldsymbol{G}(\boldsymbol{\sigma}(t)) \boldsymbol{\omega}(t), \\
& \boldsymbol{J}\dot{\boldsymbol{\omega}}(t) + \boldsymbol{\omega}(t)^{\times}\!\left(\boldsymbol{J}\boldsymbol{\omega}(t)+\boldsymbol{h}_w(t)\right)
= \boldsymbol{\tau}_{\mathrm{ref}}(t), \\
& \dot{\boldsymbol{h}}_w(t) = -\boldsymbol{\tau}_{\mathrm{ref}}(t), \qquad \boldsymbol{h}_w(0)=\boldsymbol{h}_{w,0}, \\[2pt]
& \boldsymbol{\sigma}(0)=\boldsymbol{\sigma}_0, \\
& \boldsymbol{\omega}(0)=\boldsymbol{\omega}_0, \\[2pt]
& \boldsymbol{\sigma}(T_f)=\boldsymbol{\sigma}_{\text{target}},\quad \boldsymbol{\omega}(T_f)=\boldsymbol{\omega}_{\text{target}}, \\[2pt]
& |\tau_{\mathrm{ref},i}(t)| \le (1-\gamma_u)\,\tau_{i,\max}, \quad i\in\{1,2,3\}, \\
& |h_{w,i}(t)| \le (1-\gamma_h)\,h_{w,i,\max}, \quad i\in\{1,2,3\}.
\end{aligned}
\label{eq:OCP_slack}
\end{equation}


Here $\lambda>0$ weights control effort and smoothness, while $\gamma_u,\gamma_h\in(0,1)$ tighten the
actuator bounds to reserve headroom for inner-loop tracking and disturbance rejection. 

The OCP is solved offline using the Gauss pseudospectral method GPOPS-II \cite{patterson_gpops-ii_2014},
producing nodal torques $\{\boldsymbol{\tau}_i\}_{i=0}^N$ at nodes $\{t_i\}_{i=0}^N$. A continuous-time
torque command is obtained via piecewise-linear interpolation:
\begin{equation}
\tilde{\boldsymbol{\tau}}(t) = \boldsymbol{\tau}_i + \frac{t-t_i}{t_{i+1}-t_i}\left(\boldsymbol{\tau}_{i+1}-\boldsymbol{\tau}_i\right),
\quad t\in[t_i,t_{i+1}],\ i=0,\dots,N-1.
\label{eq:linear_interpolation}
\end{equation}
The interpolated torque $\tilde{\boldsymbol{\tau}}(t)$ is then integrated through
\eqref{eq:attitude_kinematics}--\eqref{eq:attitude_dynamics} to reconstruct a dynamically consistent
reference trajectory $(\boldsymbol{\sigma}_d(t),\boldsymbol{\omega}_d(t))$. 
Since each component satisfies $|\tau_{i}(t_i)|\le(1-\gamma_u)\tau_{i,\max}$ and $|\tau_{i}(t_{i+1})|\le(1-\gamma_u)\tau_{i,\max}$,
the piecewise-linear interpolation implies $|\tilde{\tau}_{i}(t)|\le(1-\gamma_u)\tau_{i,\max}$ for all $t\in[t_i,t_{i+1}]$.


\begin{remark}
The outer-loop planner is not posed as a standalone contribution in optimal control. Its role is to embed
the dual saturation limits and terminal boundary conditions into reference generation, providing a
physically realizable, time-anchored trajectory for the inner-loop prescribed-time tracker.
\end{remark}

\begin{remark}
Assumption~1 implies the existence of a minimum feasible maneuver time $T_{f,\min}$ under the dual saturation
constraints. A conservative estimate can be obtained via a bang-bang (time-optimal) construction under
component-wise torque limits; an explicit procedure is provided in Appendix~\ref{app:bangbang}.
\end{remark}


\subsection{Inner-Loop Nonsingular Prescribed-Time Tracking Control}

To facilitate a nonsingular prescribed-accuracy prescribed-time design, this section introduces the following sliding surface that couples 
the attitude and angular-velocity tracking errors.

Define the sliding surface as
\begin{equation}
\boldsymbol{s} = \boldsymbol{\omega}_e + \boldsymbol{Q}(\boldsymbol{\sigma}_e),
\end{equation}
where
\begin{equation}
\boldsymbol{Q}(\boldsymbol{\sigma}_e) 
= c_1 k_1 \frac{\boldsymbol{\sigma}_e}{(\|\boldsymbol{\sigma}_e\| + \varepsilon_1)^\eta}
+ c_1 k_2 \frac{\boldsymbol{\sigma}_e}{(\|\boldsymbol{\sigma}_e\| + \varepsilon_1)^{-\eta}},
\end{equation}
with constants $0 < \eta < 1$, $\varepsilon_1 > 0$,
\[
c_1 = \frac{\pi}{\eta T_{p1}}, \quad
k_1 = (1 + \alpha_1) 2^{1 + \frac{3}{2}\eta}, \quad
k_2 = 2^{1 - \frac{\eta}{2}},
\]
where $T_{p1} > 0$ and $\alpha_1 > 0$.

\begin{theorem}
If the sliding surface $\boldsymbol{s}(t)$ converges to the bounded set $\|\boldsymbol{s}(t)\| \le s_{\max}$.
Then the attitude error $\boldsymbol{\sigma}_e(t)$ will converge within time $T_{p1}$ to the set
\begin{equation}
\|\boldsymbol{\sigma}_e(t)\| \le \max\left\{ \varepsilon_{\delta_1},\ \varepsilon_1 \right\}
\end{equation}
where
\begin{equation}
\varepsilon_{\delta_1} := \sqrt{2} \left( \frac{\delta_1 \eta T_{p1}}{\pi \alpha_1} \right)^{\frac{1}{1 - \eta}}, \quad
\delta_1 := \frac{\sqrt{2}}{2} s_{\max}.
\end{equation}
\end{theorem}


\begin{proof}
Consider the Lyapunov function candidate
\begin{equation}
V_1 = \frac{1}{2}\|\boldsymbol{\sigma}_e\|^2.
\end{equation}
For $t$ such that $\|\boldsymbol{s}(t)\| \le s_{\max}$, differentiating along the error kinematics $\dot{\boldsymbol{\sigma}}_e = \boldsymbol{G}(\boldsymbol{\sigma}_e)\boldsymbol{\omega}_e$ and substituting $\boldsymbol{\omega}_e = \boldsymbol{s} - \boldsymbol{Q}(\boldsymbol{\sigma}_e)$ yields
\begin{align}
\dot{V}_1 
&= \boldsymbol{\sigma}_e^\top \boldsymbol{G}(\boldsymbol{\sigma}_e)(\boldsymbol{s} - \boldsymbol{Q}(\boldsymbol{\sigma}_e)) \nonumber \\
&\le \|\boldsymbol{G}(\boldsymbol{\sigma}_e)\|\,\|\boldsymbol{\sigma}_e\|\,s_{\max}
- \boldsymbol{\sigma}_e^\top \boldsymbol{G}(\boldsymbol{\sigma}_e) \boldsymbol{Q}(\boldsymbol{\sigma}_e).
\end{align}
From \eqref{eq:G_norm_bound}, $\|\boldsymbol{G}(\boldsymbol{\sigma}_e)\| < \tfrac{1}{2}$ for $\|\boldsymbol{\sigma}_e\| < 1$. Assuming $\|\boldsymbol{\sigma}_e\| \ge \varepsilon_1$, Lemma~\ref{lem:norm_ineq} gives
\[
\frac{\|\boldsymbol{\sigma}_e\|^2}{(\|\boldsymbol{\sigma}_e\| + \varepsilon_1)^\eta} \ge \frac{1}{2^\eta} \|\boldsymbol{\sigma}_e\|^{2 - \eta},
\quad
\frac{\|\boldsymbol{\sigma}_e\|^2}{(\|\boldsymbol{\sigma}_e\| + \varepsilon_1)^{-\eta}} \ge \|\boldsymbol{\sigma}_e\|^{2 + \eta}.
\]
Thus,
\begin{align}
\dot{V}_1 
&\le \frac{1}{2} \|\boldsymbol{\sigma}_e\| s_{\max}
- \frac{c_1}{4} \left[
k_1 \frac{\|\boldsymbol{\sigma}_e\|^2}{(\|\boldsymbol{\sigma}_e\| + \varepsilon_1)^\eta}
+ k_2 \frac{\|\boldsymbol{\sigma}_e\|^2}{(\|\boldsymbol{\sigma}_e\| + \varepsilon_1)^{-\eta}}
\right] \nonumber \\
&\le \frac{\sqrt{2}}{2} V_1^{1/2} s_{\max}
- \frac{c_1}{4} \left[
k_1 2^{-\eta} (2 V_1)^{1 - \eta/2}
+ k_2 (2 V_1)^{1 + \eta/2}
\right].
\end{align}
Substituting $c_1 = \frac{\pi}{\eta T_{p1}}$, $k_1 = (1 + \alpha_1) 2^{1 + \frac{3}{2}\eta}$, and $k_2 = 2^{1 - \frac{\eta}{2}}$ yields
\begin{equation}
\dot{V}_1 \le \delta_1 V_1^{1/2} - \frac{\pi}{\eta T_{p1}} \left[ (1 + \alpha_1) V_1^{1 - \eta/2} + V_1^{1 + \eta/2} \right].
\end{equation}
This satisfies the condition of Theorem~\ref{thm:practical_PT}. Hence, $V_1(t)$ converges within time $T_{p1}$ to
\begin{equation}
V_1 \le \frac{1}{2} \varepsilon_{\delta_1}^2.
\end{equation}
Since $V_1 = \tfrac{1}{2} \|\boldsymbol{\sigma}_e\|^2$, it follows that 
$\|\boldsymbol{\sigma}_e(t)\| \le \varepsilon_{\delta_1}$. Moreover, 
the regularization ensures that the set $\{ \|\boldsymbol{\sigma}_e\| \le \varepsilon_1 \}$ is positively invariant. 
Combining both cases yields the stated bound $\|\boldsymbol{\sigma}_e(t)\| \le \max\{\varepsilon_{\delta_1}, \varepsilon_1\}$.
\end{proof}

\begin{remark}
The parameter $\varepsilon_{\delta_1}$ depends inversely on $\alpha_1^{1/(1-\eta)}$. Therefore, for any fixed $s_{\max}$ and $T_{p1}$, one can always choose a appropriate $\alpha_1 > 0$ such that $\varepsilon_{\delta_1} < \varepsilon_1$. In this case, the ultimate convergence bound becomes $\|\boldsymbol{\sigma}_e(t)\| \le \varepsilon_1$, and the user-specified tolerance $\varepsilon_1$ fully determines the practical accuracy of the maneuver.
\end{remark}

\begin{remark}
The parameter $\varepsilon_1 > 0$ serves two purposes: (i) it eliminates the singularity in $\boldsymbol{Q}(\boldsymbol{\sigma}_e)$ as $\|\boldsymbol{\sigma}_e\| \to 0$, ensuring a smooth and bounded control law; and (ii) it provides a direct, user-adjustable interface for specifying the desired convergence tolerance. In practical applications, exact zero error is neither achievable nor necessary; $\varepsilon_1$ allows engineers to encode mission-level precision requirements directly into the controller design.
\end{remark}

Based on the above surface, construct a prescribed-time inner-loop torque command consisting of a 
computable feedforward cancellation term and a dominant prescribed-time stabilizing term.
The resulting inner-loop control law is given by
\begin{equation}
\boldsymbol{\tau} = -\boldsymbol{f}(t) - \boldsymbol{J}\dot{\boldsymbol{Q}}(\boldsymbol{\sigma}_e) + \boldsymbol{\tau}_c,
\label{eq:tau_law_feasibility}
\end{equation}
where $\boldsymbol{f}(t)$ is the known nonlinear term defined in \eqref{eq:simplified_error_dynamics}, and the prescribed-time dominant term $\boldsymbol{\tau}_c$ is given by
\begin{equation}
\boldsymbol{\tau}_c = -c_2 \left[
k_3 \frac{\boldsymbol{s}}{(\|\boldsymbol{s}\| + \varepsilon_2)^\eta}
+ k_4 \frac{\boldsymbol{s}}{(\|\boldsymbol{s}\| + \varepsilon_2)^{-\eta}}
\right],
\end{equation}
with design parameters $0 < \eta < 1$, $\varepsilon_2 > 0$, and
\[
c_2 = \frac{\pi}{\eta T_{p2}}, \quad
k_3 = (1 + \alpha_2) 2^{-1 + \frac{3}{2}\eta} \lambda_{\max}(\boldsymbol{J})^{1 - \frac{\eta}{2}}, \quad
k_4 = 2^{-1 - \frac{\eta}{2}} \lambda_{\max}(\boldsymbol{J})^{1 + \frac{\eta}{2}}.
\]
Here, $T_{p2} > 0$ is the prescribed convergence time for the sliding surface, and $\alpha_2 > 0$ is a tunable gain.

Substituting the control law into the attitude error dynamics \eqref{eq:simplified_error_dynamics} and using the definition of $\boldsymbol{s}$, the closed-loop sliding surface dynamics are obtained. Left-multiplying by $\boldsymbol{J}$ yields
\begin{equation}
\boldsymbol{J} \dot{\boldsymbol{s}} = \boldsymbol{\tau}_c + \boldsymbol{d}(t)
\label{eqL:closed_dynamics_s}
\end{equation}

\begin{theorem}
Under Assumption~\ref{ass:disturbance_bound}, the sliding variable $\boldsymbol{s}(t)$ converges to the bounded set
\begin{equation}
\|\boldsymbol{s}(t)\| \le \max\left\{ \varepsilon_{\delta_2},\ \varepsilon_2 \right\},
\end{equation}
where
\begin{equation}
\varepsilon_{\delta_2} := \sqrt{2} \left( \frac{\delta_2 \eta T_{p2}}{\pi \alpha_2} \right)^{\frac{1}{1 - \eta}}, \quad
\delta_2 := d_{\max} \sqrt{\frac{2}{\lambda_{\min}(\boldsymbol{J})}}.
\end{equation}
\end{theorem}

\begin{proof}
Consider the Lyapunov function
\begin{equation}
V_2 = \frac{1}{2} \boldsymbol{s}^\top \boldsymbol{J} \boldsymbol{s}.
\end{equation}
Its time derivative along the closed-loop dynamics \eqref{eqL:closed_dynamics_s} is
\begin{equation}
\dot{V}_2 = \boldsymbol{s}^\top \boldsymbol{\tau}_c + \boldsymbol{s}^\top \boldsymbol{d}(t).
\end{equation}
For the disturbance term, using $\|\boldsymbol{s}\| \le \sqrt{2 V_2 / \lambda_{\min}(\boldsymbol{J})}$ yields
\begin{equation}
\boldsymbol{s}^\top \boldsymbol{d}(t) \le \|\boldsymbol{s}\| d_{\max}
\le \delta_2 V_2^{1/2}, \quad \delta_2 := d_{\max} \sqrt{\frac{2}{\lambda_{\min}(\boldsymbol{J})}}
\end{equation}
For the control term, assuming $\|\boldsymbol{s}\| \ge \varepsilon_2$ , Lemma~\ref{lem:norm_ineq} gives
\[
\frac{\|\boldsymbol{s}\|^2}{(\|\boldsymbol{s}\| + \varepsilon_2)^\eta} \ge \frac{1}{2^\eta} \|\boldsymbol{s}\|^{2 - \eta},
\quad
\frac{\|\boldsymbol{s}\|^2}{(\|\boldsymbol{s}\| + \varepsilon_2)^{-\eta}} \ge \|\boldsymbol{s}\|^{2 + \eta}.
\]
Thus,
\begin{align}
\boldsymbol{s}^\top \boldsymbol{\tau}_c
&= -c_2 \left[
k_3 \frac{\|\boldsymbol{s}\|^2}{(\|\boldsymbol{s}\| + \varepsilon_2)^\eta}
+ k_4 \frac{\|\boldsymbol{s}\|^2}{(\|\boldsymbol{s}\| + \varepsilon_2)^{-\eta}}
\right]
\nonumber \\
&\le -c_2 \left[
k_3 2^{-\eta} \|\boldsymbol{s}\|^{2 - \eta}
+ k_4 \|\boldsymbol{s}\|^{2 + \eta}
\right]
\nonumber \\
&\le -\frac{\pi}{\eta T_{p2}} \left[
(1 + \alpha_2) V_2^{1 - \eta/2} + V_2^{1 + \eta/2}
\right],
\end{align}
where in the last step we used $c_2 = \pi/(\eta T_{p2})$, $k_3 = (1 + \alpha_2) 2^{-1 + \frac{3}{2}\eta} \lambda_{\max}(\boldsymbol{J})^{1 - \eta/2}$,
$k_4 = 2^{-1 - \frac{\eta}{2}} \lambda_{\max}(\boldsymbol{J})^{1 + \eta/2}$, and the fact that
$\|\boldsymbol{s}\|^2 \le 2 V_2 / \lambda_{\min}(\boldsymbol{J}) \le 2 V_2$ (since $\lambda_{\min}(\boldsymbol{J}) \le 1$ w.l.o.g. or absorb constants into gain design).

Combining both terms, we obtain
\begin{equation}
\dot{V}_2 \le \delta_2 V_2^{1/2} - \frac{\pi}{\eta T_{p2}} \left[ (1 + \alpha_2) V_2^{1 - \frac{\eta}{2}} + V_2^{1 + \frac{\eta}{2}} \right],
\end{equation}
which satisfies the condition of Theorem~\ref{thm:practical_PT}. Hence, $V_2(t)$ converges to the set
$V_2 \le \frac{1}{2} \varepsilon_{\delta_2}^2$ in finite time and remains therein. Since
$\|\boldsymbol{s}\| \le \sqrt{2 V_2 / \lambda_{\min}(\boldsymbol{J})} \le \sqrt{2} \sqrt{V_2}$,
it follows that $\|\boldsymbol{s}(t)\| \le \varepsilon_{\delta_2}$.
Moreover, the regularization ensures that the set $\{ \|\boldsymbol{s}\| \le \varepsilon_2 \}$ is positively invariant.
Combining both cases yields the stated bound $\|\boldsymbol{s}(t)\| \le \max\{\varepsilon_{\delta_2}, \varepsilon_2\}$.
\end{proof}

\begin{remark}
By selecting an appropriate value of $\alpha_2 > 0$ in conjunction with $\varepsilon_2$, the designer can ensure $\varepsilon_{\delta_2} \le \varepsilon_2$, so that the ultimate bound on the sliding surface is governed by the user-specified tolerance $\varepsilon_2$. This avoids unnecessarily large control gains while achieving the desired steady-state tracking precision.
\end{remark}

\begin{remark}
The inner-loop convergence occurs in two stages. First, Theorem~3 guarantees that
$\|\boldsymbol{s}(t)\|\le \max\{\varepsilon_{\delta_2},\varepsilon_2\}$ for all $t\ge T_{p2}$.
Set $s_{\max}:=\max\{\varepsilon_{\delta_2},\varepsilon_2\}$.
Then Theorem~2 implies that $\boldsymbol{\sigma}_e(t)$ enters
$\|\boldsymbol{\sigma}_e\|\le \max\{\varepsilon_{\delta_1},\varepsilon_1\}$ for all $t\ge T_{p2}+T_{p1}$.
Therefore, it suffices to require $T_{p1}+T_{p2}<T_f$ to ensure the tracking errors settle before the terminal time.
\end{remark}


The computable feedforward term $\boldsymbol{f}(t)$ in \eqref{eq:simplified_error_dynamics} can be decomposed as
\begin{equation}
\boldsymbol{f}(t)=\boldsymbol{\tau}_{\mathrm{ref}}(t)+\boldsymbol{\Delta}_f(t),
\label{eq:f_split}
\end{equation}
where $\boldsymbol{\Delta}_f(t):=\boldsymbol{f}(t)-\boldsymbol{\tau}_{\mathrm{ref}}(t)$ and its component-wise bound
is given in Appendix~\ref{app:Deltaf_bound}. The planner enforces
$|\tau_{\mathrm{ref},i}(t)|\le (1-\gamma_u)\tau_{i,\max}$ on $[0,T_f]$.

\begin{theorem}
\label{thm:feasibility_axis_compact2}
Consider the inner-loop control law \eqref{eq:tau_law_feasibility}.
Assume that along the closed-loop trajectory over $[0,T_f]$ the tracking errors remain inside the prescribed-accuracy tube,
namely,$\|\boldsymbol{\sigma}_e(t)\|\le \varepsilon_1,
\|\boldsymbol{s}(t)\|\le \varepsilon_2$ for all $t\in[0,T_f]$.

If the safety margin $\gamma_u$ and prescribed-accuracy set $(\varepsilon_1,\varepsilon_2)$ are selected such that
\begin{equation}
\Delta_{f,i,\max}(\varepsilon_1,\varepsilon_2)
+\overline{(J\dot Q)}_i(\varepsilon_1,\varepsilon_2)
+\bar\tau_c(\varepsilon_2)
\le \gamma_u\,\tau_{i,\max},\qquad i=1,2,3,
\label{eq:thm4_axis_condition_compact2}
\end{equation}
where $\Delta_{f,i,\max}(\varepsilon_1,\varepsilon_2)$,
$\overline{(J\dot Q)}_i(\varepsilon_1,\varepsilon_2)$, and $\bar\tau_c(\varepsilon_2)$
are explicit bounds derived in the proof.
Then the actuator torques satisfy $|\tau_i(t)|\le \tau_{i,\max}$ for all $t\in[0,T_f]$ and all $i=1,2,3$.
\end{theorem}


\begin{proof}
By the tube assumption, $\|\boldsymbol{\sigma}_e(t)\| \le \varepsilon_1$ and $\|\boldsymbol{s}(t)\| \le \varepsilon_2$ for all $t \in [0, T_f]$.
From $\boldsymbol{s} = \boldsymbol{\omega}_e + \boldsymbol{Q}(\boldsymbol{\sigma}_e)$, it follows that
\begin{equation}
\|\boldsymbol{\omega}_e(t)\| \le \varepsilon_2 + \bar Q
\label{eq:wemax}
\end{equation}
where $\bar Q$ is given in \eqref{eq:omegae_barQ}.
\begin{equation}
\bar Q:=\sup_{\|\boldsymbol{\sigma}_e\|\le \varepsilon_1}\|\boldsymbol{Q}(\boldsymbol{\sigma}_e)\|
\le c_1\Big(k_1\,2^{-\eta}\varepsilon_1^{1-\eta}+k_2\,2^{\eta}\varepsilon_1^{1+\eta}\Big).
\label{eq:omegae_barQ}
\end{equation}


From \eqref{eq:tau_law_feasibility} and \eqref{eq:f_split}, for each axis $i$,
\begin{equation}
|\tau_i|
\le |\tau_{\mathrm{ref},i}|+|\Delta_{f,i}|+\big|(\boldsymbol{J}\dot{\boldsymbol{Q}})_i\big|+|\tau_{c,i}|.
\label{eq:tau_axis_triangle}
\end{equation}
By the planner tightening, $|\tau_{\mathrm{ref},i}|\le (1-\gamma_u)\tau_{i,\max}$.


\emph{(i) Bound on $|\Delta_{f,i}|$.}
Appendix~\ref{app:Deltaf_bound} yields, for $i=1,2,3$,
\begin{equation}
|\Delta_{f,i}(t)|
\le (\|\boldsymbol{\omega}_e(t)\|+\|\boldsymbol{\omega}_d(t)\|)\,H_{\max}
+\mu_i\,\|\boldsymbol{\omega}_e(t)\|\,\|\boldsymbol{\omega}_d(t)\|
+4\mu_i\,\|\boldsymbol{\sigma}_e(t)\|\,\|\dot{\boldsymbol{\omega}}_d(t)\|,
\label{eq:Deltaf_i_pointwise_inproof}
\end{equation}
where $\mu_i:=\sum_{j=1}^{3}|J_{ij}|$ and $H_{\max}$ is given in Assumption~3.
Using \eqref{eq:omegae_barQ} and \eqref{eq:ref_bounded}, we obtain
\begin{equation}
|\Delta_{f,i}(t)|\le \Delta_{f,i,\max}(\varepsilon_1,\varepsilon_2),
\label{eq:Deltaf_i_envelope_inproof}
\end{equation}
with
\begin{equation}
\Delta_{f,i,\max}(\varepsilon_1,\varepsilon_2):=
(\varepsilon_2+\bar Q+\omega_{d,\max})\,H_{\max}
+\mu_i(\varepsilon_2+\bar Q)\,\omega_{d,\max}
+4\mu_i\,\varepsilon_1\,\dot\omega_{d,\max}.
\label{eq:Deltafimax_eps_inproof}
\end{equation}

\emph{(ii) Bound on $|\tau_{c,i}|$.}
Since $\|\boldsymbol{s}(t)\|\le \varepsilon_2$,
\begin{equation}
|\tau_{c,i}(t)|
\le \bar\tau_c(\varepsilon_2),
\qquad
\bar\tau_c(\varepsilon_2):=
c_2\Big(k_3\,2^{-\eta}\varepsilon_2^{1-\eta}+k_4\,2^{\eta}\varepsilon_2^{1+\eta}\Big).
\label{eq:thm4_tauc_bar_compact2}
\end{equation}

\emph{(iii) Bound on $\big|(\boldsymbol{J}\dot{\boldsymbol{Q}})_i\big|$.}
By $\dot{\boldsymbol{Q}}=\frac{\partial \boldsymbol{Q}}{\partial \boldsymbol{\sigma}_e}\boldsymbol{G}(\boldsymbol{\sigma}_e)\boldsymbol{\omega}_e$ and \eqref{eq:G_norm_bound},
\[
\|\dot{\boldsymbol{Q}}(t)\|
\le \Big\|\frac{\partial \boldsymbol{Q}}{\partial \boldsymbol{\sigma}_e}\Big\|
\frac{1+\varepsilon_1^2}{4}\,(\varepsilon_2+\bar Q).
\]
Moreover, for $\rho:=\|\boldsymbol{\sigma}_e\|\le\varepsilon_1$, a convenient Jacobian envelope is
\begin{equation}
\Big\|\frac{\partial \boldsymbol{Q}}{\partial \boldsymbol{\sigma}_e}\Big\|
\le \Psi(\varepsilon_1),
\qquad
\Psi(\rho):=
c_1\Big[
k_1(\rho + \varepsilon_1)^{-\eta}
+ k_2(\rho + \varepsilon_1)^{\eta}
+ \rho\big(
k_1\eta(\rho + \varepsilon_1)^{-\eta-1}
+ k_2\eta(\rho + \varepsilon_1)^{\eta-1}
\big)
\Big].
\label{eq:Psi_def_thm4_local}
\end{equation}
Let $\mu_i:=\sum_{j=1}^{3}|J_{ij}|$, then $|(\boldsymbol{J}\boldsymbol{v})_i|\le\mu_i\|\boldsymbol{v}\|$, and hence
\begin{equation}
\big|(\boldsymbol{J}\dot{\boldsymbol{Q}}(t))_i\big|
\le \overline{(J\dot Q)}_i(\varepsilon_1,\varepsilon_2),
\qquad
\overline{(J\dot Q)}_i:=
\mu_i\,\Psi(\varepsilon_1)\,\frac{1+\varepsilon_1^2}{4}\,(\varepsilon_2+\bar Q).
\label{eq:thm4_JQdot_bar_compact2}
\end{equation}

Finally, substituting the bounds (i)--(iii) into \eqref{eq:tau_axis_triangle} yields
\[
|\tau_i(t)|
\le (1-\gamma_u)\tau_{i,\max}
+\Delta_{f,i,\max}(\varepsilon_1,\varepsilon_2)
+\overline{(J\dot Q)}_i(\varepsilon_1,\varepsilon_2)
+\bar\tau_c(\varepsilon_2).
\]
Thus, if \eqref{eq:thm4_axis_condition_compact2} holds, then $|\tau_i(t)|\le \tau_{i,\max}$ for all $t\in[0,T_f]$ and all $i$.
\end{proof}

\begin{theorem}
\label{thm:feasibility_momentum}
Consider the inner-loop control law \eqref{eq:tau_law_feasibility}.
Assume that along the closed-loop trajectory over $[0,T_f]$ the tracking errors remain inside the prescribed-accuracy tube,
namely, $\|\boldsymbol{\sigma}_e(t)\|\le \varepsilon_1$,
$\|\boldsymbol{s}(t)\|\le \varepsilon_2$ for all $t\in[0,T_f]$.

If the safety margin $\gamma_h$ and prescribed-accuracy set $(\varepsilon_1,\varepsilon_2)$ are selected such that
\begin{equation}
H_{\max} + \mu_i (\varepsilon_2 + \bar Q) \le \gamma_h\, h_{w,i,\max},\qquad i=1,2,3,
\label{eq:thm5_axis_condition}
\end{equation}
where $\bar Q$ is defined in \eqref{eq:omegae_barQ} and $\mu_i:=\sum_{j=1}^{3}|J_{ij}|$,
then the wheel angular momenta satisfy $|h_{w,i}(t)|\le h_{w,i,\max}$ for all $t\in[0,T_f]$ and all $i=1,2,3$.
\end{theorem}

\begin{proof}
By the tube assumption , the angular-velocity tracking error $\|\boldsymbol{\omega}_e\|$ satisfies \eqref{eq:wemax} for all $t \in [0, T_f]$.

From $\boldsymbol{h}_w = \boldsymbol{H} - \boldsymbol{J}\boldsymbol{\omega}$ and the reference momentum $\boldsymbol{h}_{w,d} = \boldsymbol{H}_0 - \boldsymbol{J}\boldsymbol{\omega}_d$,
the actual wheel momentum can be written as
\[
\boldsymbol{h}_w = \boldsymbol{h}_{w,d} + (\boldsymbol{H} - \boldsymbol{H}_0) - \boldsymbol{J}\boldsymbol{\omega}_e.
\]
Taking norms and applying the triangle inequality yields, for each axis $i$,
\begin{align}
  |h_{w,i}| &\le |h_{w,d,i}| + \|\boldsymbol{H} - \boldsymbol{H}_0\| + \mu_i (\varepsilon_2 + \bar Q) \nonumber \\
  &\le (1 - \gamma_h) h_{w,i,\max} + H_{\max} + \mu_i (\varepsilon_2 + \bar Q)
\end{align}
where we used the OCP constraint $|h_{w,d,i}| \le (1 - \gamma_h) h_{w,i,\max}$ and Assumption~\ref{ass:H_budget}.

Under condition \eqref{eq:thm5_axis_condition}, the right-hand side is bounded by $h_{w,i,\max}$,
which completes the proof.
\end{proof}

\begin{remark}
The closed-loop system is initialized on the planned reference trajectory, i.e.,
$\boldsymbol{\sigma}(0)=\boldsymbol{\sigma}_d(0)$ and $\boldsymbol{\omega}(0)=\boldsymbol{\omega}_d(0)$, hence
$\boldsymbol{\sigma}_e(0)=\boldsymbol{0}$ and $\boldsymbol{s}(0)=\boldsymbol{0}$.
By Theorems~2--3, the regularized neighborhoods $\{\|\boldsymbol{s}\|\le\varepsilon_2\}$ and
$\{\|\boldsymbol{\sigma}_e\|\le\varepsilon_1\}$ are forward invariant; once entered, the trajectories do not leave.
Therefore, in this paper the condition in Theorem~\ref{thm:feasibility_axis_compact2} and 
Theorem~\ref{thm:feasibility_momentum} holds over $[0,T_f]$.
\end{remark}


\subsection{Overall Design Procedure of the EPTPAC Framework}
\label{overall_design}

This subsection summarizes a mission-driven synthesis procedure for the proposed EPTPAC framework.

\begin{enumerate}
\item \textbf{Mission inputs and physical limits.}
Specify the terminal time \(T_f>0\) and the mission accuracy requirement \(\varepsilon_{\text{mission}}>0\).
The inertia \(\boldsymbol{J}\), wheel limits \(\tau_{\max},h_{w,\max}\), and disturbance bound \(d_{\max}\) (Assumption~2) are known.

\item \textbf{Margins and feasibility.}
Choose tightening margins \(\gamma_u,\gamma_h\in(0,1)\) to reserve actuation headroom, e.g. \(\gamma_u \gtrsim 2d_{\max}/u_{\max}\) (similarly for \(\gamma_h\)).
Verify \(T_f\ge T_{f,\min}\) (Assumption~1) via time-optimal analysis or a numerical feasibility test.

\item \textbf{Accuracy/time allocation and \(\varepsilon_2\) scaling.}
Set \(\varepsilon_1:=\varepsilon_{\text{mission}}\), pick \(\eta\in(0,1)\), and allocate \(T_{p1},T_{p2}>0\) such that \(T_{p1}+T_{p2}<T_f\).
Since the sliding surface is \( \boldsymbol{s}=\boldsymbol{\omega}_e+\boldsymbol{Q}(\boldsymbol{\sigma}_e)\) and \(\|\boldsymbol{Q}\|=\mathcal{O}(\|\boldsymbol{\sigma}_e\|^{1-\eta})\), 
set
\begin{equation}
\varepsilon_2 = c\,\varepsilon_1^{1-\eta},\qquad c>0.
\label{eq:eps2_scaling_short}
\end{equation}

\item \textbf{Gain selection.}
Choose \(\alpha_1,\alpha_2>0\) such that \(\varepsilon_{\delta_1}\le\varepsilon_1\) and \(\varepsilon_{\delta_2}\le\varepsilon_2\) in Theorems~2--3.
With \(s_{\max}=\varepsilon_2\) and \(\delta_2=d_{\max}\sqrt{2/\lambda_{\min}(\boldsymbol{J})}\), sufficient lower bounds are
\begin{equation}
\alpha_1 \ge \alpha_{1,\min}:=\frac{\eta T_{p1}}{\pi}\,2^{-\eta/2}\,\frac{\varepsilon_2}{\varepsilon_1^{1-\eta}},\qquad
\alpha_2 \ge \alpha_{2,\min}:=\frac{\delta_2\,\eta\,T_{p2}}{\pi}\left(\frac{\sqrt{2}}{\varepsilon_2}\right)^{1-\eta}.
\label{eq:alpha_min_short}
\end{equation}

\item \textbf{Verify saturation feasibility.}
Using the planned reference and the resulting $\varepsilon_\sigma$ from Step~3,
verify the sufficient non-saturation condition in Theorem~\ref{thm:feasibility_axis_compact2}.
If violated, increase $T_f$ (or $T_{p1},T_{p2}$), relax $\varepsilon_1$,
or replan with larger margins $\gamma_u,\gamma_h$ and/or a smoother reference.
\end{enumerate}

It is worth noting that the proposed synthesis is not tuning-intensive. 
In practice, the only parameter that typically requires adjustment is the exponent \(\eta\), which mainly trades off smoothness and transient aggressiveness. 
The remaining quantities are either specified by the mission and hardware (\(T_f,\varepsilon_{\text{mission}},u_{\max},h_{w,\max},d_{\max},\boldsymbol{J}\)) 
or chosen as conservative robustness margins (\(\gamma_u,\gamma_h,\kappa\)). 
Once these are set, the rest of the controller parameters (including \(\varepsilon_2\) and \(\alpha_1,\alpha_2\)) follow automatically from the closed-form rules above.

\begin{figure}[htbp]
    \centering
    \includegraphics[height=0.5\textheight]{fig/overall_design.pdf}
    \caption{overall design procedure of the EPTPAC framework.}
    \label{fig:overall_design}
\end{figure}

\begin{remark}[Recommended balanced synthesis]
While \eqref{eq:eps2_scaling_short} provides a physically meaningful scaling between \(\varepsilon_2\) and \(\varepsilon_1\), 
the constant \(c\) may be selected automatically to avoid unbalanced gains. 
A convenient choice is to enforce \(\alpha_1=\alpha_2=\kappa\,\alpha\) with a safety factor \(\kappa>1\), and determine \(\varepsilon_2\) and \(\alpha\) by solving \(\alpha_{1,\min}=\alpha_{2,\min}\) in \eqref{eq:alpha_min_short}. This yields
\begin{equation}
\varepsilon_2
=\left(\frac{\delta_2\,\eta\,T_{p2}}{\eta T_{p1}}\,2^{1/2}\right)^{\frac{1}{2-\eta}}
\varepsilon_1^{\frac{1-\eta}{2-\eta}},\qquad
\alpha=\frac{\eta T_{p1}}{\pi}\,2^{-\eta/2}\,\frac{\varepsilon_2}{\varepsilon_1^{1-\eta}},
\end{equation}
followed by \(\alpha_1=\alpha_2=\kappa\,\alpha\). This balanced rule yields a unique and reproducible parameter set from mission inputs and physical limits.
\end{remark}


% =========================
% Section 4 (before Case 1): concise intro + unified setup table + planning paragraph
% =========================

\section{Simulation Results}
\label{section:simulation}

Numerical simulations validate the proposed EPTPAC framework under dual reaction-wheel saturation.
Table~\ref{tab:sim_setup_all} summarizes the spacecraft parameters, actuator limits, disturbance model, and boundary conditions. 
Theoretical guarantees apply over the mission interval $[0,T_f]$. The simulation horizon is extended to $T_{\mathrm{sim}}>T_f$ to observe post-maneuver behavior, with the reference held constant at the target state for $t\ge T_f$.

\begin{table}[htbp]
\centering
\caption{Common simulation settings and boundary conditions}
\label{tab:sim_setup_all}
\begin{tabular}{@{}l l@{}}
\toprule
\textbf{Symbol} & \textbf{Value} \\
\midrule
$\boldsymbol{J}$ &
$\mathrm{diag}([200,\ 250,\ 300])~\si{kg\cdot m^2}$ \\
$\tau_{i,\max}$ &
$\SI{0.2}{N\cdot m}$ \\
$h_{w,i,\max}$ &
$\SI{4.0}{kg\cdot m^2/s}$ \\
$\boldsymbol{d}(t)$ &
$0.01\,[\sin(2t),\ \cos(t),\ \cos(t+2)]^\top~\si{N\cdot m}$ \\
$d_{\max}$ &
$\sqrt{3} \times \SI{0.01}{N\cdot m}$ \\
$\boldsymbol{\sigma}(0)$ &
$[0.2,\ 0.3,\ -0.3]^\top$ \\
$\boldsymbol{\sigma}_{\text{target}}$ &
$[0,\ 0,\ 0]^\top$ \\
$\boldsymbol{\omega}(0)$ &
$[0,\ 0,\ 0]^\top~\si{rad/s}$ \\
$\boldsymbol{\omega}_{\text{target}}$ &
$[0,\ 0,\ 0]^\top~\si{rad/s}$ \\
$\boldsymbol{h}_w(0)$ &
$[0,\ 0,\ 0]^\top~\si{kg\cdot m^2/s}$ \\
\bottomrule
\end{tabular}
\end{table}

The outer-loop reference $(\boldsymbol{\sigma}_d,\boldsymbol{\omega}_d)$ is generated offline by solving the constrained OCP
\eqref{eq:OCP_slack} using GPOPS-II with $\lambda=0.1$ and
$N=20$ nodes. The nodal torques are converted to a continuous-time command via
piecewise-linear interpolation \eqref{eq:linear_interpolation}.

Four comprehensive cases are studied to validate different aspects of the framework: Case~1 validates baseline performance with dual-constraint compliance; Case~2 assesses exact terminal-time convergence across different $T_f$; Case~3 demonstrates accuracy tunability; and Case~4 compares against an existing prescribed-time controller.

\subsection{Case 1: Baseline Maneuver}
This case validates the baseline performance of the EPTPAC framework, demonstrating exact terminal-time convergence and strict compliance with dual actuator constraints.
The terminal time is set to $T_f=\SI{120}{s}$ with mission-level attitude accuracy $\varepsilon_1=10^{-5}$.
The inner-loop convergence times are allocated as $T_{p2}=\SI{15}{s}$ and $T_{p1}=\SI{100}{s}$, ensuring $T_{p1}+T_{p2}=\SI{115}{s}<T_f$ to provide temporal margin.
Design parameters include the prescribed-time exponent $\eta=0.2$ and safety margins $\gamma_u=\gamma_h=0.1$ for actuator headroom.

Following the synthesis procedure in Section~\ref{overall_design} with balancing factor $\kappa=1.2$, the inner-loop parameters are computed as $\varepsilon_2=8.0213\times10^{-5}$ and $\alpha_1=\alpha_2=5.7174$.
The practical convergence bounds from Theorems~2--3 are $\varepsilon_{\delta_1}=7.9620\times10^{-6}$ and $\varepsilon_{\delta_2}=6.3865\times10^{-5}$, both satisfying $\varepsilon_{\delta_i}<\varepsilon_i$.
Thus, the ultimate tracking accuracy is governed by the user-specified tolerances $(\varepsilon_1,\varepsilon_2)$.

Figure~\ref{fig:tracking_performance} illustrates the reference tracking performance. The spacecraft attitude in MRPs and angular velocity closely follows the planned trajectory, with the reference reaching the target state exactly at $t=T_f$ by construction.

Figure~\ref{fig:convergence_behavior} displays the error convergence profiles. Due to consistent initialization ($\boldsymbol{\sigma}(0)=\boldsymbol{\sigma}_d(0)$, $\boldsymbol{\omega}(0)=\boldsymbol{\omega}_d(0)$), all tracking errors start at zero and remain within prescribed neighborhoods throughout the maneuver.


Figure~\ref{fig:constraint_satisfaction} demonstrates physical realizability under dual saturation. Both torque components and wheel momentum remain strictly within their prescribed limits throughout $[0,T_f]$, validating the constraint-aware trajectory planning and bounded inner-loop corrections.

\begin{figure}[htbp]
    \centering
    \begin{minipage}[b]{0.48\linewidth}\centering
        \includegraphics[width=\linewidth]{fig/Case1_Results_AttitudeComparison_tf120.0.pdf}\\[2pt]
        \footnotesize (a) Attitude MRPs
    \end{minipage}\hfill
    \begin{minipage}[b]{0.48\linewidth}\centering
        \includegraphics[width=\linewidth]{fig/Case1_Results_omega_AttitudeComparison_tf120.0.pdf}\\[2pt]
        \footnotesize (b) Angular velocity
    \end{minipage}
    \caption{Case~1: Reference tracking in MRPs and angular rates.}
    \label{fig:tracking_performance}
\end{figure}

\begin{figure}[htbp]
    \centering
    \begin{minipage}[b]{0.32\linewidth}\centering
        \includegraphics[width=\linewidth]{fig/Case1_Results_AttitudeError_tf120.0.pdf}\\[2pt]
        \footnotesize (a) \( \boldsymbol{\sigma}_e \)
    \end{minipage}\hfill
    \begin{minipage}[b]{0.32\linewidth}\centering
        \includegraphics[width=\linewidth]{fig/Case1_Results_AngularRateError_tf120.0.pdf}\\[2pt]
        \footnotesize (b) \( \boldsymbol{\omega}_e \)
    \end{minipage}\hfill
    \begin{minipage}[b]{0.32\linewidth}\centering
        \includegraphics[width=\linewidth]{fig/Case1_Results_sError_tf120.0.pdf}\\[2pt]
        \footnotesize (c) \( \boldsymbol{s} \)
    \end{minipage}
    \caption{Case~1: Errors responses}
    \label{fig:convergence_behavior}
\end{figure}

\begin{figure}[htbp]
    \centering
    \begin{minipage}[b]{0.48\linewidth}\centering
        \includegraphics[width=\linewidth]{fig/Case1_Results_ControlTorque_tf120.0.pdf}\\[2pt]
        \footnotesize (a) \( \boldsymbol{\tau} \)
    \end{minipage}\hfill
    \begin{minipage}[b]{0.48\linewidth}\centering
        \includegraphics[width=\linewidth]{fig/Case1_Results_WheelMomentum_tf120.0.pdf}\\[2pt]
        \footnotesize (b) \( \boldsymbol{h}_w \)
    \end{minipage}
    \caption{Case~1: Torque and wheel momentum.}
    \label{fig:constraint_satisfaction}
\end{figure}

To assess robustness against beyond-bound transient perturbations, a severe disturbance injection is applied during the first second:
\begin{equation}
\boldsymbol{d}(t)=
\begin{cases}
0.05\,[1,\,1,\,1]^\top~\si{N\cdot m}, & 0\le t<1,\\
\boldsymbol{d}_{\mathrm{base}}(t), & t\ge 1,
\end{cases}
\label{eq:case1_step_disturbance}
\end{equation}
where $\boldsymbol{d}_{\mathrm{base}}(t)$ denotes the nominal bounded disturbance from Table~\ref{tab:sim_setup_all}.
The injected disturbance magnitude $0.05~\si{N\cdot m}$ exceeds the design bound $d_{\max} \approx 0.0173~\si{N\cdot m}$, potentially driving the trajectory outside the guaranteed practical convergence neighborhoods of Assumption~2.

Despite this severe transient, Figs.~\ref{fig:case1_step_convergence}--\ref{fig:case1_step_constraints} demonstrate robust recovery. Upon disturbance removal at $t=1~\si{s}$, the tracking errors rapidly return to their prescribed neighborhoods within the scheduled convergence times $T_{p1}+T_{p2}=\SI{115}{s}$.

Critically, both torque and wheel momentum remain strictly within their saturation limits throughout the entire maneuver, including during the disturbance period, validating the framework's robustness margins and constraint-aware design.

\begin{figure}[htbp]
    \centering
    \begin{minipage}[b]{0.32\linewidth}\centering
        \includegraphics[width=\linewidth]{fig/Case1p_Results_AttitudeError_tf120.0.pdf}\\[2pt]
        \footnotesize (a) \( \boldsymbol{\sigma}_e \)
    \end{minipage}\hfill
    \begin{minipage}[b]{0.32\linewidth}\centering
        \includegraphics[width=\linewidth]{fig/Case1p_Results_AngularRateError_tf120.0.pdf}\\[2pt]
        \footnotesize (b) \( \boldsymbol{\omega}_e \)
    \end{minipage}\hfill
    \begin{minipage}[b]{0.32\linewidth}\centering
        \includegraphics[width=\linewidth]{fig/Case1p_Results_sError_tf120.0.pdf}\\[2pt]
        \footnotesize (c) \( \boldsymbol{s} \)
    \end{minipage}
    \caption{Case~1: Error convergence under beyond-bound transient disturbance.}
    \label{fig:case1_step_convergence}
\end{figure}

\begin{figure}[htbp]
    \centering
    \begin{minipage}[b]{0.48\linewidth}\centering
        \includegraphics[width=\linewidth]{fig/Case1p_Results_ControlTorque_tf120.0.pdf}\\[2pt]
        \footnotesize (a) \( \boldsymbol{\tau} \)
    \end{minipage}\hfill
    \begin{minipage}[b]{0.48\linewidth}\centering
        \includegraphics[width=\linewidth]{fig/Case1p_Results_WheelMomentum_tf120.0.pdf}\\[2pt]
        \footnotesize (b) \( \boldsymbol{h}_w \)
    \end{minipage}
    \caption{Case~1: Torque and wheel momentum under the beyond-bound transient disturbance.}
    \label{fig:case1_step_constraints}
\end{figure}

This case demonstrates time-anchored convergence with dual-constraint compliance and rapid recovery under beyond-bound disturbances.




\subsection{Case 2: Time-Flexibility Assessment}

Before presenting Case~2, we clarify notation. The tracking error $\boldsymbol{\sigma}_e$ is defined relative to the reference $\boldsymbol{\sigma}_d(t)$.
To highlight the prescribed terminal-time tracking relative to the fixed target, we additionally report the target-pointing MRP error
$\boldsymbol{\sigma}_{e,\mathrm{tar}}$, defined between the current attitude $\boldsymbol{\sigma}(t)$ and the constant target $\boldsymbol{\sigma}_{\mathrm{target}}$.

This case verifies that the framework ensures exact terminal-time convergence across different terminal times while maintaining feasibility
under dual actuator constraints. For all $T_f\in\{\SI{110}{s},\,\SI{120}{s},\,\SI{130}{s},\,\SI{140}{s}\}$, the closed-loop system reaches
the target at the commanded time with prescribed tracking accuracy, while control torque and wheel momentum remain within limits.

The study uses the same settings as Case~1, with $\varepsilon_1=10^{-5}$, $\eta=0.2$, $\gamma_u=\gamma_h=0.1$, and time allocation 
$T_{p2}=\SI{15}{s}$, $T_{p1}=T_f-\SI{20}{s}$ (so $T_{p1}+T_{p2}=T_f-\SI{5}{s}<T_f$). For each $T_f$, the outer-loop OCP \eqref{eq:OCP_slack} is solved to generate a dual-constraint-feasible reference, and inner-loop parameters are synthesized using the balanced rule with $\kappa=1.2$.

\begin{table}[htbp]
\centering
\caption{Case~2: Synthesized parameters for terminal-time sweep}
\label{tab:case2_params}
\begin{tabular}{@{}c c c c c@{}}
\toprule
$T_f$ & $T_{p1}$ & $T_{p2}$ & $\varepsilon_2$ & $\alpha_1=\alpha_2$ \\
\midrule
$\SI{110}{s}$ & $\SI{90}{s}$  & $\SI{15}{s}$ & $8.505\times10^{-5}$ & $5.456$ \\
$\SI{120}{s}$ & $\SI{100}{s}$ & $\SI{15}{s}$ & $8.021\times10^{-5}$ & $5.717$ \\
$\SI{130}{s}$ & $\SI{110}{s}$ & $\SI{15}{s}$ & $7.608\times10^{-5}$ & $5.965$ \\
$\SI{140}{s}$ & $\SI{120}{s}$ & $\SI{15}{s}$ & $7.249\times10^{-5}$ & $6.200$ \\
\bottomrule
\end{tabular}
\end{table}

Figures~\ref{fig:Case2_qe_xyz}--\ref{fig:Case2_h_xyz} display the responses. Figure~\ref{fig:Case2_qe_xyz} shows that the target-pointing
error components decay into prescribed neighborhoods before their respective terminal times, confirming exact terminal-time convergence. 
Figures~\ref{fig:Case2_tau_xyz} and \ref{fig:Case2_h_xyz} reveal the expected trade-off: shorter $T_f$ demands more aggressive actuation with higher torque and faster wheel-momentum accumulation, yet both remain within hardware limits due to constraint-aware planning.

\begin{figure}[htbp]
  \centering
  \begin{minipage}[b]{0.32\linewidth}\centering
    \includegraphics[width=\linewidth]{fig/qe_x_comparison.png}\\[2pt]
    \footnotesize (a) \( \sigma_{e1, \mathrm{tar}} \)
  \end{minipage}\hfill
  \begin{minipage}[b]{0.32\linewidth}\centering
    \includegraphics[width=\linewidth]{fig/qe_y_comparison.png}\\[2pt]
    \footnotesize (b) \( \sigma_{e2, \mathrm{tar}} \)
  \end{minipage}\hfill
  \begin{minipage}[b]{0.32\linewidth}\centering
    \includegraphics[width=\linewidth]{fig/qe_z_comparison.png}\\[2pt]
    \footnotesize (c) \( \sigma_{e3, \mathrm{tar}} \)
  \end{minipage}
  \caption{Case~2: MRP error components for different $T_f$}
  \label{fig:Case2_qe_xyz}
\end{figure}

\begin{figure}[htbp]
  \centering
  \begin{minipage}[b]{0.32\linewidth}\centering
    \includegraphics[width=\linewidth]{fig/tau_x_comparison.png}\\[2pt]
    \footnotesize (a) \( \tau_x \)
  \end{minipage}\hfill
  \begin{minipage}[b]{0.32\linewidth}\centering
    \includegraphics[width=\linewidth]{fig/tau_y_comparison.png}\\[2pt]
    \footnotesize (b) \( \tau_y \)
  \end{minipage}\hfill
  \begin{minipage}[b]{0.32\linewidth}\centering
    \includegraphics[width=\linewidth]{fig/tau_z_comparison.png}\\[2pt]
    \footnotesize (c) \( \tau_z \)
  \end{minipage}
  \caption{Case 2: Control torque components for different $T_f$}
  \label{fig:Case2_tau_xyz}
\end{figure}

\begin{figure}[htbp]
  \centering
  \begin{minipage}[b]{0.32\linewidth}\centering
    \includegraphics[width=\linewidth]{fig/h_x_comparison.png}\\[2pt]
    \footnotesize (a) \( h_x \)
  \end{minipage}\hfill
  \begin{minipage}[b]{0.32\linewidth}\centering
    \includegraphics[width=\linewidth]{fig/h_y_comparison.png}\\[2pt]
    \footnotesize (b) \( h_y \)
  \end{minipage}\hfill
  \begin{minipage}[b]{0.32\linewidth}\centering
    \includegraphics[width=\linewidth]{fig/h_z_comparison.png}\\[2pt]
    \footnotesize (c) \( h_z \)
  \end{minipage}
  \caption{Case 2: Wheel momentum components for different $T_f$}
  \label{fig:Case2_h_xyz}
\end{figure}

The table \ref{tab:case2_convergence_time} reveals that the actual convergence time is marginally earlier than the prescribed terminal time $T_f$, with a discrepancy of approximately $0.2$--$0.3$ seconds. 
This early entry is attributed to the conservative design of the accuracy tolerance $\varepsilon_1 = 10^{-5}$, which ensures robust performance margins under disturbances and modeling errors. 
Since the early arrival is minimal and the tracking error remains well within the prescribed accuracy bound, the framework successfully achieves practical exact-time convergence while maintaining safety margins in the presence of uncertainties.

In summary, Case~2 confirms that the proposed framework provides a practical interface to schedule the maneuver at user-specified
terminal times. By re-planning the dual-constraint-feasible reference on $[0,T_f]$ and re-synthesizing the tracking parameters
accordingly, the closed-loop system ensures exact terminal-time convergence across a wide range of $T_f$ values. The results also reveal the expected
trade-off: shorter $T_f$ leads to higher torque demand and faster wheel-momentum accumulation, while all runs remain realizable under
the prescribed actuator limits due to the constraint-aware co-design.

\subsection{Case 3: Precision-sweep assessment under identical conditions}
\label{sec:case3_precision_sweep}

This case demonstrates prescribed-accuracy tunability under the same operating condition as Case~1, with the terminal time fixed at
$T_f=\SI{120}{s}$. Set $\eta=0.2$, $\gamma_u=\gamma_h=0.1$, $T_{p2}=\SI{15}{s}$, and $T_{p1}=\SI{100}{s}$, and sweep the
prescribed attitude tolerance as $\varepsilon_1\in\{10^{-4},\,10^{-5},\,10^{-6},\,10^{-7}\}$. For each $\varepsilon_1$, the remaining
inner-loop parameters are synthesized using the balanced rule in Section~\ref{overall_design} with $\kappa=1.2$, producing
$(\varepsilon_2,\alpha_1,\alpha_2)$ together with the practical bounds $(\varepsilon_{\delta_1},\varepsilon_{\delta_2})$ from
Theorems~2--3. The resulting values are reported in Table~\ref{tab:case3_params}.

\begin{table}[htbp]
\centering
\caption{Case~3: Synthesized parameters for accuracy sweep}
\label{tab:case3_params}
\begin{tabular}{@{}c c c c c@{}}
\toprule
$\varepsilon_1$ & $\varepsilon_2$ & $\alpha_1=\alpha_2$ & $\varepsilon_{\delta_1}$ & $\varepsilon_{\delta_2}$ \\
\midrule
$1\times10^{-4}$ & $2.232\times10^{-4}$ & $2.521$ & $7.962\times10^{-5}$ & $1.777\times10^{-4}$ \\
$1\times10^{-5}$ & $8.021\times10^{-5}$ & $5.717$ & $7.962\times10^{-6}$ & $6.387\times10^{-5}$ \\
$1\times10^{-6}$ & $2.883\times10^{-5}$ & $12.965$ & $7.962\times10^{-7}$ & $2.295\times10^{-5}$ \\
$1\times10^{-7}$ & $1.036\times10^{-5}$ & $29.398$ & $7.962\times10^{-8}$ & $8.249\times10^{-6}$ \\
\bottomrule
\end{tabular}
\end{table}

The log-scale responses in Fig.~\ref{fig:case3_precision_compare} show that all runs achieve their prescribed accuracy:
$\|\boldsymbol{\sigma}_e(t)\|$ decreases below the threshold $\varepsilon_1$ and remains therein, while $\|\boldsymbol{s}(t)\|$ is regulated below $\varepsilon_2$. 
This confirms the framework provides a direct interface for tuning mission accuracy through $\varepsilon_1$.

\begin{figure}[htbp]
    \centering
    \begin{minipage}[b]{0.48\linewidth}\centering
        \includegraphics[width=\linewidth]{fig/CaseSweep_MRPErrorNorm_log10.pdf}\\[2pt]
        \footnotesize (a) \(\log_{10}(\|\boldsymbol{\sigma}_e\|)\) with \(\varepsilon_1\) thresholds
    \end{minipage}\hfill
    \begin{minipage}[b]{0.48\linewidth}\centering
        \includegraphics[width=\linewidth]{fig/CaseSweep_SlidingNorm_log10.pdf}\\[2pt]
        \footnotesize (b) \(\log_{10}(\|\boldsymbol{s}\|)\) with \(\varepsilon_2\) thresholds
    \end{minipage}
    \caption{Case~3: Log-scale comparison of $\|\boldsymbol{\sigma}_e\|$ and $\|\boldsymbol{s}\|$ for different $\varepsilon_1$}
    \label{fig:case3_precision_compare}
\end{figure}

As $\varepsilon_1$ tightens from $10^{-4}$ to $10^{-7}$, the achieved accuracy decreases and induced sliding tolerance $\varepsilon_2$ also decreases.
Correspondingly, gains $\alpha_1$ and $\alpha_2$ increase substantially (Table~\ref{tab:case3_params}), raising control activity. 
In practice, $\varepsilon_1$ should be selected to match sensor accuracy and actuation resources; pursuing excessively tight tolerances is unnecessary and may degrade robustness margins.

\begin{table}[htbp]
\centering
\caption{Case~2: First entry time of $\|\boldsymbol{\sigma}_{e,\mathrm{tar}}\|$ into prescribed accuracy neighborhoods}
\label{tab:case2_convergence_time}
\begin{tabular}{@{}c S[table-format=3.2] S[table-format=3.2]@{}}
\toprule
{$T_f$ (\si{s})} & {First entry into $10^{-5}$ (\si{s})} & {First entry into $10^{-6}$ (\si{s})} \\
\midrule
110 & 109.75 & 109.94 \\
120 & 119.76 & 119.93 \\
130 & 129.75 & 129.94 \\
140 & 139.75 & 139.94 \\
\bottomrule
\end{tabular}
\end{table}

In summary, Case~3 confirms that the framework systematically enforces prescribed accuracy levels through $\varepsilon_1$. 
Tightening $\varepsilon_1$ reduces tracking residual but requires larger gains and higher control effort. Thus, $\varepsilon_1$ should be selected based on achievable sensing accuracy and actuation margins.


\subsection{Case 4: Comparison with Existing Prescribed-Time Saturated Control}

To demonstrate the superiority of the proposed framework, Case~4 compares against a representative predefined-time saturated controller from \cite{xu_distributed_2022}. 
Two scenarios evaluate prescribed terminal-time tracking performance and robustness to dual saturation.

\subsubsection{Case 4.1: Prescribed terminal-time tracking under varying margins}

This scenario evaluates prescribed terminal-time tracking under two commands: an ample-margin case ($T_f=300~\si{s}$) and a tight-margin case ($T_f=150~\si{s}$). 
Only torque saturation ($\tau_{i,\max}=0.2~\si{N\cdot m}$) is enforced. Results are shown in Fig.~\ref{fig:case4_1_comparison}.

\begin{figure}[htbp]
    \centering
    % 第一行:300s 情况下 对应图片后缀为 total500.0.pdf
    \subfloat[Attitude error ($T_f = 300\,\si{s}$)]{\includegraphics[width=0.48\linewidth]{fig/PTC_compare_Attitude__t_total400.0.pdf}\label{fig:err300}}
    \hfill
    \subfloat[Control torque ($T_f = 300\,\si{s}$)]{\includegraphics[width=0.48\linewidth]{fig/PTC_compare_ControlTorque_t_total400.0.pdf}\label{fig:tau300}}
    \\
    % 第二行:150s 情况下 对应图片后缀为 total300.0.pdf
    \subfloat[Attitude error ($T_f = 150\,\si{s}$)]{\includegraphics[width=0.48\linewidth]{fig/PTC_compare_Attitude__t_total300.0.pdf}\label{fig:err150}}
    \hfill
    \subfloat[Control torque ($T_f = 150\,\si{s}$)]{\includegraphics[width=0.48\linewidth]{fig/PTC_compare_ControlTorque_t_total300.0.pdf}\label{fig:tau150}}
    
    \caption{Case 4.1: Performance comparison under torque-only saturation.}
    \label{fig:case4_1_comparison}
\end{figure}

For the ample-margin case ($T_f=300~\si{s}$), Figs.~\ref{fig:err300} and \ref{fig:tau300} reveal that the baseline exhibits significant conservatism: 
tracking error converges around $150~\si{s}$, far earlier than the prescribed $300~\si{s}$. 
This premature convergence induces unnecessary control effort and wastes momentum resources—undesirable for long-duration missions or synchronized operations.

For the tight-margin case ($T_f=150~\si{s}$), the baseline's limitations become acute. As shown in Figs.~\ref{fig:err150} and \ref{fig:tau150}, 
high-gain feedback combined with torque limits destroys the nominal prescribed-time behavior. 
The spacecraft fails to arrive within 150 seconds, with convergence delayed to approximately 230 seconds.

In contrast, EPTPAC ensures convergence exactly at the prescribed times with smooth, optimized profiles. This comparison demonstrates that existing prescribed-time methods, while theoretically sound without constraints, fail to maintain reliable prescribed terminal-time tracking when saturations are enforced. EPTPAC resolves this by embedding physical limits into the planning layer.

\subsubsection{Case 4.2: Dual saturation (torque and wheel momentum)}

This scenario evaluates robustness under stringent dual saturation constraints with $T_f=300~\si{s}$. Results are shown in Fig.~\ref{fig:case4_2_comparison}.

\begin{figure}[htbp]
    \centering
    % 第一行:动态响应对比
    \subfloat[Attitude error norm]{\includegraphics[width=0.48\linewidth]{fig/PTC_compare_Attitude__t_total500.0.pdf}\label{fig:err_dual}}
    \hfill
    \subfloat[Angular velocity error norm]{\includegraphics[width=0.48\linewidth]{fig/PTC_compare_omega__t_total500.0.pdf}\label{fig:omega_dual}}
    \\
    % 第二行:执行器状态对比
    \subfloat[Control torque component ($\tau_x$)]{\includegraphics[width=0.48\linewidth]{fig/PTC_compare_ControlTorque_t_total500.0.pdf}\label{fig:tau_dual}}
    \hfill
    \subfloat[Wheel angular momentum ($h_w$)]{\includegraphics[width=0.48\linewidth]{fig/PTC_compare_WheelMomentum_t_total500.0.pdf}\label{fig:hw_dual}}
    
    \caption{Case 4.2: Performance comparison under dual saturation}
    \label{fig:case4_2_comparison}
\end{figure}

The simulation results reveal a significant performance gap between the two frameworks 
under dual saturation. As shown in Fig.~\ref{fig:case4_2_comparison}(d), the baseline controller, 
lacking a predictive awareness of momentum storage, demands excessive control effort early 
in the maneuver. This causes the wheel momentum to hit its hard ceiling of $4.0~\si{kg\cdot m^2/s}$ at 
approximately $t = 120~\si{s}$, effectively stripping the spacecraft of its control authority. 
During this saturation interval, the control torque is forced to zero or remains ineffective 
(Fig.~\ref{fig:case4_2_comparison}(c)), leading to the uncontrolled "drift" observed in the attitude 
and angular velocity errors (Figs.~\ref{fig:case4_2_comparison}(a) and (b)). Consequently, the baseline's convergence is delayed to $410~\si{s}$, far exceeding the 300-second requirement.

In contrast, the EPTPAC framework successfully navigates these mission-critical constraints. 
The outer-loop trajectory planner generates a reference that strategically allocates torque to 
prevent the wheels from ever reaching the momentum ceiling. As a result, the inner-loop tracking 
errors remain small and well-behaved throughout the mission window. By tracking this 
constraint-feasible reference, the spacecraft reaches the target state with the prescribed accuracy 
exactly at $t = 300~\si{s}$. This outcome demonstrates that the integration of optimal planning and 
prescribed-time control is indispensable for ensuring the physical realizability of timed maneuvers 
in the presence of complex dual-saturation effects.

In summary, Case~4 shows that enforcing actuator saturation can destroy the nominal terminal-time behavior of predefined-time controllers
derived under unconstrained assumptions, whereas EPTPAC ensures exact terminal-time convergence by co-designing a saturation-feasible reference and a
bounded prescribed-time tracker.


\section{Conclusion}
\label{section:conclusion}

This paper proposed a EPTPAC framework for reaction-wheel spacecraft that must reach a commanded attitude \emph{exactly} at a user-specified terminal time while strictly satisfying both torque and wheel momentum limits. The approach combines a constraint-aware, time-anchored trajectory planner with a nonsingular prescribed-time tracking controller, yielding explicit practical prescribed-time accuracy bounds and a sufficient feasibility condition that relates actuator limits, margins, prescribed-time parameters, and admissible initial errors.

Simulation results demonstrated exact-time arrival with guaranteed constraint satisfaction under significant disturbances, consistent performance across multiple prescribed terminal times, and predictable accuracy changes under a precision sweep. Comparisons with a conventional prescribed-time controller under enforced actuator constraints further illustrated that saturation can destroy the nominal prescribed-time behavior and significantly prolong settling, underscoring the necessity of constraint-aware co-design for physically realizable timed maneuvers.

Future work will investigate reduced-complexity/online planning, robustness to model uncertainty, and integrated momentum-management strategies for long-duration operations.

\appendix
\section{A Bang-Bang Estimate of $T_{f,\min}$}
\label{app:bangbang}

This appendix provides a bang-bang estimate of the minimum feasible maneuver time under component-wise torque and wheel-momentum limits.
It is used as a conservative screening rule for selecting $T_f$ in Assumption~1.

Let $\boldsymbol{\sigma}_e(0)$ denote the initial attitude error in MRPs. The associated principal rotation angle $\theta\in[0,\pi]$ and axis
$\boldsymbol{e}=[e_1,e_2,e_3]^T$ are
\begin{equation}
\theta = 4\arctan\!\big(\|\boldsymbol{\sigma}_e(0)\|\big),\qquad
\boldsymbol{e} =
\begin{cases}
\dfrac{\boldsymbol{\sigma}_e(0)}{\|\boldsymbol{\sigma}_e(0)\|}, & \boldsymbol{\sigma}_e(0)\neq \boldsymbol{0},\\[6pt]
\text{any unit vector}, & \boldsymbol{\sigma}_e(0)=\boldsymbol{0}.
\end{cases}
\label{eq:app_axis_angle}
\end{equation}
Define the effective inertia $J_e:=\boldsymbol{e}^T\boldsymbol{J}\boldsymbol{e}$. With $|\tau_i|\le \tau_{i,\max}$, the maximum torque projection along $\boldsymbol{e}$ is
\begin{equation}
\tau_{e,\max}=\max_{|\tau_i|\le \tau_{i,\max}} \boldsymbol{e}^T\boldsymbol{\tau}
=\sum_{i=1}^{3} |e_i|\,\tau_{i,\max}.
\label{eq:app_tau_emax}
\end{equation}

Assume $\boldsymbol{h}_w(0)=\boldsymbol{0}$ and $|h_{w,i}|\le h_{w,i,\max}$. Under the bang-bang allocation $\tau_i=\mathrm{sgn}(e_i)\tau_{i,\max}$, the maximum duration before wheel-momentum saturation is
\begin{equation}
t_h:=\min_{i\in\{1,2,3\}} \frac{h_{w,i,\max}}{\tau_{i,\max}}.
\label{eq:app_th}
\end{equation}

For a nominal rest-to-rest rotation about $\boldsymbol{e}$, the peak rate and the corresponding no-coast angle are
\begin{equation}
\omega_{\max}=\frac{\tau_{e,\max}}{J_e}\,t_h,\qquad
\theta_h=\frac{\tau_{e,\max}}{J_e}\,t_h^2.
\label{eq:app_omega_theta_h}
\end{equation}
The resulting minimum-time estimate is
\begin{equation}
T_{f,\min}\approx
\begin{cases}
2\sqrt{\dfrac{J_e\,\theta}{\tau_{e,\max}}}, & \theta \le \theta_h,\\[10pt]
2t_h+\dfrac{\theta-\theta_h}{\omega_{\max}}, & \theta > \theta_h.
\end{cases}
\label{eq:app_Tmin_piecewise}
\end{equation}

% \section{Bound on the feedforward mismatch}
% \label{app:Deltaf_bound}

% Define the total angular momentum (body frame) as
% \begin{equation}
% \boldsymbol{H}(t):=\boldsymbol{J}\boldsymbol{\omega}(t)+\boldsymbol{h}_w(t),\qquad
% \boldsymbol{H}_d(t):=\boldsymbol{J}\boldsymbol{\omega}_d(t)+\boldsymbol{h}_{w,d}(t).
% \label{eq:app_H_Hd_def}
% \end{equation}
% The planned torque associated with the outer-loop reference is
% \begin{equation}
% \boldsymbol{\tau}_{\mathrm{ref}}(t)
% =\boldsymbol{J}\dot{\boldsymbol{\omega}}_d(t)+\boldsymbol{\omega}_d(t)^{\times}\boldsymbol{H}_d(t),
% \label{eq:app_tau_ref}
% \end{equation}
% and the feedforward mismatch is defined by
% \begin{equation}
% \boldsymbol{\Delta}_f(t):=\boldsymbol{f}(t)-\boldsymbol{\tau}_{\mathrm{ref}}(t).
% \label{eq:app_Deltaf_def}
% \end{equation}
% Substituting $\boldsymbol{f}(t)$ from \eqref{eq:simplified_error_dynamics} yields
% \begin{equation}
% \boldsymbol{\Delta}_f(t)=
% -\boldsymbol{\omega}(t)^{\times}\boldsymbol{H}(t)
% +\boldsymbol{J}\boldsymbol{\omega}_e(t)^{\times}\boldsymbol{R}(\boldsymbol{\sigma}_e(t))\boldsymbol{\omega}_d(t)
% +\boldsymbol{J}\big(\boldsymbol{R}(\boldsymbol{\sigma}_e(t))-\boldsymbol{I}\big)\dot{\boldsymbol{\omega}}_d(t)
% -\boldsymbol{\omega}_d(t)^{\times}\boldsymbol{H}_d(t).
% \label{eq:app_Deltaf_4terms}
% \end{equation}

% The outer-loop reference is generated under wheel-only actuation and disturbance-free dynamics.
% Moreover, Assumption~3 enforces a zero reference total angular momentum, i.e., $\boldsymbol{H}_d(t)\equiv\boldsymbol{0}$.
% Hence, the last term in \eqref{eq:app_Deltaf_4terms} vanishes and $\boldsymbol{\tau}_{\mathrm{ref}}(t)=\boldsymbol{J}\dot{\boldsymbol{\omega}}_d(t)$.

% For the attitude-dependent term, using \eqref{eq:R_sigma_e_definition} one has
% \begin{equation}
% \|\boldsymbol{R}(\boldsymbol{\sigma}_e)-\boldsymbol{I}\|
% =\frac{4\|\boldsymbol{\sigma}_e\|}{1+\|\boldsymbol{\sigma}_e\|^2}
% \le 4\|\boldsymbol{\sigma}_e\|.
% \label{eq:app_RminusI_bound}
% \end{equation}

% Let $\mu_i:=\sum_{j=1}^{3}|J_{ij}|$ so that $|(\boldsymbol{J}\boldsymbol{v})_i|\le \mu_i\|\boldsymbol{v}\|_2$.
% Using $\|\boldsymbol{a}^{\times}\boldsymbol{b}\|\le \|\boldsymbol{a}\|\,\|\boldsymbol{b}\|$, $\|\boldsymbol{R}\|=1$,
% and the mission-level momentum budget $\|\boldsymbol{H}(t)\|\le H_{\max}$ from Assumption~3, we obtain for each axis $i$:
% \begin{align}
% |\Delta_{f,i}(t)|
% \le\;& \|\boldsymbol{\omega}(t)\|\,H_{\max}
% + \mu_i\,\|\boldsymbol{\omega}_e(t)\|\,\|\boldsymbol{\omega}_d(t)\|
% + 4\mu_i\,\|\boldsymbol{\sigma}_e(t)\|\,\|\dot{\boldsymbol{\omega}}_d(t)\|.
% \label{eq:Deltaf_i_pointwise}
% \end{align}
% Furthermore, since $\|\boldsymbol{\omega}(t)\|\le \|\boldsymbol{\omega}_e(t)\|+\|\boldsymbol{\omega}_d(t)\|$, it follows that
% \begin{align}
% |\Delta_{f,i}(t)|
% \le\;&(\|\boldsymbol{\omega}_e(t)\|+\|\boldsymbol{\omega}_d(t)\|)\,H_{\max}
% + \mu_i\,\|\boldsymbol{\omega}_e(t)\|\,\|\boldsymbol{\omega}_d(t)\|
% + 4\mu_i\,\|\boldsymbol{\sigma}_e(t)\|\,\|\dot{\boldsymbol{\omega}}_d(t)\|.
% \label{eq:Deltaf_i_pointwise2}
% \end{align}

% Taking the supremum on $[0,T_f]$, for any valid bounds
% $\|\boldsymbol{\omega}_e(t)\|\le \omega_{e,\max}$,
% $\|\boldsymbol{\sigma}_e(t)\|\le \sigma_{\max}$,
% $\|\boldsymbol{\omega}_d(t)\|\le \omega_{d,\max}$,
% $\|\dot{\boldsymbol{\omega}}_d(t)\|\le \dot{\omega}_{d,\max}$,
% one may choose the component-wise envelope
% \begin{equation}
% \Delta_{f,i,\max}:=
% (\omega_{e,\max}+\omega_{d,\max})\,H_{\max}
% +\mu_i\,\omega_{e,\max}\,\omega_{d,\max}
% +4\mu_i\,\sigma_{\max}\,\dot{\omega}_{d,\max},
% \qquad i=1,2,3.
% \label{eq:app_Deltafmax_axis}
% \end{equation}










\section{Bound on the feedforward mismatch}
\label{app:Deltaf_bound}
Let $\boldsymbol{H}_d(t) := \boldsymbol{J}\boldsymbol{\omega}_d(t) + \boldsymbol{h}_{w,d}(t)$ denote the total angular momentum of the reference trajectory.
Since the reference is generated under disturbance-free dynamics and initialized with the same total momentum as the actual spacecraft (\(\boldsymbol{H}_d(0) = \boldsymbol{H}(0) =: \boldsymbol{H}_0\)),
the reference total angular momentum is conserved, i.e.,
\[
\boldsymbol{H}_d(t) \equiv \boldsymbol{H}_0, \quad \forall t \in [0, T_f].
\]

The planned torque is therefore
\begin{equation}
\boldsymbol{\tau}_{\mathrm{ref}}(t) = \boldsymbol{J}\dot{\boldsymbol{\omega}}_d(t) + \boldsymbol{\omega}_d(t)^{\times} \boldsymbol{H}_0.
\end{equation}
The feedforward mismatch is defined as $\boldsymbol{\Delta}_f(t) := \boldsymbol{f}(t) - \boldsymbol{\tau}_{\mathrm{ref}}(t)$.
Substituting $\boldsymbol{f}(t)$ from \eqref{eq:simplified_error_dynamics} yields
\begin{align}
\boldsymbol{\Delta}_f(t)
=&\; -\boldsymbol{\omega}(t)^{\times} \boldsymbol{H}(t)
+ \boldsymbol{J} \boldsymbol{\omega}_e(t)^{\times} \boldsymbol{R}(\boldsymbol{\sigma}_e(t)) \boldsymbol{\omega}_d(t)
+ \boldsymbol{J} \big( \boldsymbol{R}(\boldsymbol{\sigma}_e(t)) - \boldsymbol{I} \big) \dot{\boldsymbol{\omega}}_d(t)
- \boldsymbol{\omega}_d(t)^{\times} \boldsymbol{H}_0.
\end{align}
Grouping the momentum-dependent terms gives
\[
\boldsymbol{\Delta}_f(t)
= -\boldsymbol{\omega}(t)^{\times} \big( \boldsymbol{H}(t) - \boldsymbol{H}_0 \big)
- \boldsymbol{\omega}_e(t)^{\times} \boldsymbol{H}_0
+ \boldsymbol{J} \boldsymbol{\omega}_e(t)^{\times} \boldsymbol{R}(\boldsymbol{\sigma}_e(t)) \boldsymbol{\omega}_d(t)
+ \boldsymbol{J} \big( \boldsymbol{R}(\boldsymbol{\sigma}_e(t)) - \boldsymbol{I} \big) \dot{\boldsymbol{\omega}}_d(t).
\]

From Assumption~\ref{ass:H_budget}, the momentum drift satisfies
\begin{equation}
\|\boldsymbol{H}(t) - \boldsymbol{H}_0\| \le H_{\max}, \qquad \forall t \in [0, T_f].
\end{equation}
Using $\|\boldsymbol{a}^{\times} \boldsymbol{b}\| \le \|\boldsymbol{a}\| \|\boldsymbol{b}\|$, $\|\boldsymbol{R}(\boldsymbol{\sigma}_e)\| = 1$, and
$\|\boldsymbol{R}(\boldsymbol{\sigma}_e) - \boldsymbol{I}\| \le 4 \|\boldsymbol{\sigma}_e\|$ (valid for $\|\boldsymbol{\sigma}_e\| < 1$), each axis component of $\boldsymbol{\Delta}_f(t)$ satisfies
\begin{align}
|\Delta_{f,i}(t)|
\le\;& \|\boldsymbol{\omega}(t)\| \, H_{\max}
+ \mu_i \, \|\boldsymbol{\omega}_e(t)\| \, \|\boldsymbol{H}_0\|
+ \mu_i \, \|\boldsymbol{\omega}_e(t)\| \, \|\boldsymbol{\omega}_d(t)\|
+ 4 \mu_i \, \|\boldsymbol{\sigma}_e(t)\| \, \|\dot{\boldsymbol{\omega}}_d(t)\|,
\end{align}
where $\mu_i := \sum_{j=1}^{3} |J_{ij}|$. Since $\|\boldsymbol{H}_0\| = H_0$ and $\|\boldsymbol{\omega}(t)\| \le \|\boldsymbol{\omega}_e(t)\| + \|\boldsymbol{\omega}_d(t)\|$, we obtain
\begin{align}
|\Delta_{f,i}(t)|
\le\;& (\|\boldsymbol{\omega}_e(t)\| + \|\boldsymbol{\omega}_d(t)\|) \, H_{\max}
+ \mu_i \, \|\boldsymbol{\omega}_e(t)\| \, H_0
+ \mu_i \, \|\boldsymbol{\omega}_e(t)\| \, \|\boldsymbol{\omega}_d(t)\|
+ 4 \mu_i \, \|\boldsymbol{\sigma}_e(t)\| \, \|\dot{\boldsymbol{\omega}}_d(t)\|.
\end{align}

Taking the supremum over $[0, T_f]$ and using the bounds
$\|\boldsymbol{\omega}_e(t)\| \le \omega_{e,\max}$,
$\|\boldsymbol{\sigma}_e(t)\| \le \sigma_{\max}$,
$\|\boldsymbol{\omega}_d(t)\| \le \omega_{d,\max}$,
$\|\dot{\boldsymbol{\omega}}_d(t)\| \le \dot{\omega}_{d,\max}$,
we obtain the component-wise envelope
\begin{equation}
\Delta_{f,i,\max} :=
(\omega_{e,\max} + \omega_{d,\max}) \, H_{\max}
+ \mu_i \, \omega_{e,\max} \, H_0
+ \mu_i \, \omega_{e,\max} \, \omega_{d,\max}
+ 4 \mu_i \, \sigma_{\max} \, \dot{\omega}_{d,\max},
\qquad i = 1,2,3.
\end{equation}

\newpage
\bibliography{sample}


\end{document}
